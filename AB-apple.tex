% The contents of this file is 
% Copyright (c) 2009-  Charles R. Severance, All Righs Reserved

\chapter{Python Programming on Macintosh}
\chapter{Programação Python no Macintosh}

%In this appendix, we walk through a series of steps
%so you can run Python on Macintosh.  Since Python is
%already included in the Macintosh Operating system, we only
%need to learn how to edit Python files and run Python programs
%in the terminal window.

Neste apêndice, apresentaremos uma série de passos para que você possa 
executar o Python no Macintosh. Uma vez que Python já está incluso no
Sistema Operacional Macintosh, só precisamos aprender como editar os arquivos 
Python e executar programas Python no terminal.

%There are many approaches you can take to editing and running
%Python programs, and this is just one approach we have found
%to be very simple.

Existem várias abordagens que você pode adotar para edição e execução dos 
programas Python, e esta é somente umas das formas que encontramos, por ser 
muito simples.

%First, you need to install a programmer editor.  You
%do not want to use TextEdit or Microsoft Word to edit
%Python programs.  Programs must be in "flat-text" files
%and so you need an editor that is good at
%editing text files.

Primeiro, você precisará instalar um editor de textos. Você não vai querer 
utilizar o TextEdit ou o Microsoft Word para editar os programas Python. Os 
arquivos de programas devem estar em texto-puro então você precisará de um 
editor que é bom em editar arquivos de texto.

%Our recommended editor for Macintosh is TextWrangler which
%can be downloaded and installed from:

Recomendamos para Macintosh o editor TextWrangler que pode ser baixado e 
instalado através do seguinte endereço:

\url{http://www.barebones.com/products/TextWrangler/}

%To create a Python program, run 
%{\bf TextWrangler} from your {\bf Applications} folder.

Para criar um programa Python, execute {\bf TextWrangler} a partir da sua 
pasta de {\bf Aplicações}.

%Let's make our first Python program be:

Vamos fazer nosso primeiro programa em Python:

\beforeverb
\begin{verbatim}
print 'Hello Chuck'
\end{verbatim}
\afterverb
%
%Except that you should change it to be your name.  
%Save the file in a folder on your Desktop named 
%{\tt py4inf}.  It is best to keep your folder names short
%and not to have any spaces in your folder or file name.
%Once you have made the folder, save the file 
%into {\tt Desktop{\textbackslash}py4inf{\textbackslash}prog1.py}.
%
A única alteração que você deve fazer é referente ao nome, troque {\bf Chuck} 
pelo seu nome. Salve o arquivo em uma pasta chamada {\tt py4inf} em seu 
Desktop. É melhor manter os nomes das suas pastas pequenos e sem espaços, 
seja nas pastas ou nos nomes dos arquivos. Uma vez que você tenha criado a 
pasta, salve o arquivo dentro dela {\tt Desktop{\textbackslash}py4inf{\textbackslash}prog1.py}.

%Then run the {\bf Terminal} program.  The easiest way is to 
%press the Spotlight icon (the magnifying glass) in the upper
%right of your screen, enter ``terminal'', and launch the
%application that comes up.

Então, execute o programa através do {\bf Terminal}. A forma mais fácil de 
fazer isto é utilizando o Spotlight (a lupa) no lado superior direito da sua 
tela, e escreva ``terminal'', e execute a aplicação.

%You start in your ``home directory''.  You can see the current 
%directory by typing the {\tt pwd} command in the terminal window.

Você vai começar no seu diretório ``home''. Você pode ver o seu diretório 
corrente (que você se encontra) através digitando o comando {\tt pwd} na 
janela do terminal

\beforeverb
\begin{verbatim}
67-194-80-15:~ csev$ pwd
/Users/csev
67-194-80-15:~ csev$ 
\end{verbatim}
\afterverb
%
%you must be in the folder that contains your Python program 
%to run the program.  Use the {\tt cd} command to move to a new 
%folder and then the {\tt ls} command to list the files in the 
%folder.
%
você deve estar na pasta que contém seu arquivo de programa Python para 
executá-lo. Utilize o comando {\tt cd} para entrar em uma nova pasta, e 
depois o comando {\tt ls} para listar os arquivos na pasta.

\beforeverb
\begin{verbatim}
67-194-80-15:~ csev$ cd Desktop
67-194-80-15:Desktop csev$ cd py4inf
67-194-80-15:py4inf csev$ ls
prog1.py
67-194-80-15:py4inf csev$ 
\end{verbatim}
\afterverb
%
%To run your program, simply type the {\tt python} command followed
%by the name of your file at the command prompt and press enter.
%
Para executar o programa, digite o comando {\tt python} seguido do nome do 
seu arquivo na linha de comando e pressione enter.

\beforeverb
\begin{verbatim}
67-194-80-15:py4inf csev$ python prog1.py
Hello Chuck
67-194-80-15:py4inf csev$ 
\end{verbatim}
\afterverb
%
%You can edit the file in TextWrangler, save it, and then switch back
%to the command line and execute the program again by typing
%the file name again at the command-line prompt.
%
Você pode editar o arquivo no TextWrangler, salvá-lo, e então voltar para 
a linha de comando e executar o programa novamente, digitando o nome do 
arquivo na linha de comando.

%If you get confused in the command-line window, just close it
%and open a new one.

Se você ficar confuso com a linha de comando, apenas feche-a e abra uma nova 
janela.

%Hint: You can also press the ``up-arrow'' in the command line to 
%scroll back and run a previously entered command again.

Dica: Você também pode pressionar a ``seta para cima'' na linha de comando 
para executar um comando executado anteriormente.

%You should also look in the preferences for TextWrangler and set it 
%to expand tab characters to be four spaces.  It will save you lots
%of effort looking for indentation errors.

Você também deve verificar as preferências do TextWrangler e definir para que 
o caractere {\tt tab} seja substituido por quatro espaço. Isto evitará 
perder tempo procurando por erros de indentação.

%You can also find further information on editing and running 
%Python programs at \url{www.py4inf.com}.

Você também pode encontrar maiores informações sobre como editar e executar 
programas Python no endereço \url{www.py4inf.com}.

