% The contents of this file is 
% Copyright (c) 2009- Charles R. Severance, All Righs Reserved

\chapter{Prefácio}
%\chapter{Preface}

\section*{Python para Informática: Adaptação de um livro aberto}
%\section*{Python for Informatics: Remixing an Open Book}

É muito comum que acadêmicos, em sua profissão, necessitem publicar
continuamente materiais ou artigos quando querem criar algo do zero. Este livro
é um experimento em não partir da estaca zero, mas sim ``remixar'' o livro
entitulado \emph{Think Python: How to Think Like a Computer Scientist} escrito
por Allen B. Downey, Jeff Elkner e outros.

%It is quite natural for academics who are continuously told to 
%``publish or perish'' to want to always create something from scratch
%that is their own fresh creation.   This book is an 
%experiment in not starting from scratch, but instead ``remixing''
%the book titled
%\emph{Think Python: How to Think Like
%a Computer Scientist}
%written by Allen B. Downey, Jeff Elkner, and others.

Em dezembro de 2009, quando estava me preparando para ministrar a disciplina
{\bf SI502 - Programação para Redes} na Universidade de Michigan para o
quinto semestre e decidi que era hora de escrever um livro de Python focado
em explorar dados ao invés de entender algoritmos e abstrações. Minha meta em
SI502 é ensinar pessoas a terem habilidades na manipulação de dados para a
vida usando Python.  Alguns dos meus estudantes planejavam se tornarem
profissionais em programação de computadores.  Ao invés disso, eles escolheram
ser bibliotecários, gerentes, advogados, biólogos, economistas, etc., e
preferiram utilizar habilmente a tecnologia nas áreas de suas escolhas.

%In December of 2009, I was preparing to teach
%{\bf SI502 - Networked Programming} at the University of Michigan
%for the fifth semester in a row and decided it was time
%to write a Python textbook that focused on exploring data
%instead of understanding algorithms and abstractions.
%My goal in SI502 is to teach people lifelong data handling 
%skills using Python.  Few of my
%students were planning to be professional 
%computer programmers.  Instead, they
%planned to be librarians, managers, lawyers, biologists, economists, etc., 
%who happened to want to skillfully use technology in their chosen field.

Eu nunca consegui encontrar o livro perfeito sobre Python que fosse orientado a dados
para utilizar no meu curso, então eu comecei a escrever o meu próprio. Com
muita sorte, em uma reunião eventual três semanas antes de eu começar a
escrever o meu novo livro do zero, em um descanso no feriado, Dr. Atul Prakash
me mostrou o \emph{Think Python} livro que ele tinha usado para ministrar seu
curso de Python naquele semestre. Era um texto muito bem escrito sobre Ciência
da Computação com foco em explicações diretas e simples de se aprender.

%I never seemed to find the perfect data-oriented Python
%book for my course, so I set out 
%to write just such a book.  Luckily at a faculty meeting three weeks
%before I was about to start my new book from scratch over 
%the holiday break, 
%Dr. Atul Prakash showed me the \emph{Think Python} book which he had
%used to teach his Python course that semester.  
%It is a well-written Computer Science text with a focus on 
%short, direct explanations and ease of learning. 

Toda a estrutura do livro foi alterada, visando a resolução de problemas de
análise de dados de um modo tão simples e rápido quanto possível, acrescido
de uma série de exemplos práticos e exercícios sobre análise de dados desde
o início.

%The overall book structure
%has been changed to get to doing data analysis problems as quickly as
%possible and have a series of running examples and exercises 
%about data analysis from the very beginning.  

Os capítulos 2--10 são similares ao livro \emph{Think Python} mas precisaram
de muitas alterações. Exemplos com numeração e exercícios foram substituídos
por exercícios orientados a dados. Tópicos foram apresentados na ordem
necessária para construir soluções sofisticadas em análise de dados. Alguns
tópicos tais como {\tt try} e {\tt except} foram movidos mais para o final e
apresentados como parte do capítulo de condicionais. Funções foram necessárias
para simplificar a complexidade na manipulação dos programas introduzidos
anteriormente nas primeiras lições em abstração. Quase todas as funções
definidas pelo usuário foram removidas dos exemplos do código e exercícios,
com exceção do Capítulo 4. A palavra ``recursão''\footnote{Com exceção,
	naturalmente, desta linha.} não aparece no livro inteiro.

%Chapters 2--10 are similar to the \emph{Think Python} book,
%but there have been major changes. Number-oriented examples and
%exercises have been replaced with data-oriented exercises.
%Topics are presented in the order needed to build increasingly
%sophisticated data analysis solutions. Some topics like {\tt try} and
%{\tt except} are pulled forward and presented as part of the chapter
%on conditionals.  Functions are given very light treatment until 
%they are needed to handle program complexity rather than introduced 
%as an early lesson in abstraction.  Nearly all user-defined functions
%have been removed from the example code and exercises outside of Chapter 4.
%The word ``recursion''\footnote{Except, of course, for this line.}
%does not appear in the book at all.

Nos capítulos 1 e 11--16, todo o material é novo, focado em exemplos reais de
uso e exemplos simples de Python para análise de dados incluindo expressões
regulares para busca e transformação, automação de tarefas no seu computador,
recuperação de dados na internet, extração de dados de páginas web,
utilização de {\it web services}, transformação de dados em XML para JSON, e a
criação e utilização de bancos de dados utilizando SQL (Linguagem estruturada
de consulta em bancos de dados).

%In chapters 1 and 11--16, all of the material is brand new, focusing
%on real-world uses and simple examples of Python for data analysis 
%including regular expressions for searching and parsing, 
%automating tasks on your computer, retrieving data across 
%the network, scraping web pages for data, 
%using web services, parsing XML and JSON data, and creating 
%and using databases using Structured Query Language.

O último objetivo de todas estas alterações é a mudança de foco, de Ciência da
Computação para uma Informática que inclui somente tópicos que podem ser
utilizados em uma turma de primeira viagem (iniciantes) que podem ser úteis
mesmo se a escolha deles não for seguir uma carreira profissional em
programação de computadores.

%The ultimate goal of all of these changes is a shift from a 
%Computer Science to an Informatics
%focus is to only include topics into a first technology 
%class that can be useful even if one chooses not to 
%become a professional programmer.

Estudantes que acharem este livro interessante e quiserem se aprofundar devem
olhar o livro de Allen B. Downey's \emph{Think Python}. Porque há muita
sinergia entre os dois livros, estudantes irão rapidamente desenvolver
habilidades na área com a técnica de programação e o pensamento em algoritmos,
que são cobertos em \emph{Think Python}. Os dois livros possuem um estilo de
escrita similar, é possível mover-se para o livro \emph{Think Python} com o
mínimo de esforço.

%Students who find this book interesting and want to further explore
%should look at Allen B. Downey's \emph{Think Python} book.  Because there
%is a lot of overlap between the two books,
%students will quickly pick up skills in the additional
%areas of technical programming and algorithmic thinking 
%that are covered in \emph{Think Python}.
%And given that the books have a similar writing style, they should be 
%able to move quickly through \emph{Think Python} with a minimum of effort.

\index{Creative Commons License}
\index{CC-BY-SA}
\index{BY-SA}
Com os direitos autorais de \emph{Think Python}, Allen me deu permissão para
trocar a licença do livro em relação ao livro no qual este material é baseado
de GNU Licença Livre de Documentação para a mais recente Creative Commons
Attribution --- Licença de compartilhamento sem ciência do autor. Esta
baseia-se na documentação aberta de licenças mudando da GFDL para a CC-BY-SA
(i.e., Wikipedia). Usando a licença CC-BY-SA, os mantenedores deste livro
recomendam fortemente a tradição ``copyleft'' que incentiva os novos autores
a reutilizarem este material da forma como considerarem adequada.

%\index{Creative Commons License}
%\index{CC-BY-SA}
%\index{BY-SA}
%As the copyright holder of \emph{Think Python},
%Allen has given me permission to change the book's license 
%on the material from his book that remains in this book
%from the
%GNU Free Documentation License 
%to the more recent
%Creative Commons Attribution --- Share Alike
%license.
%This follows a general shift in open documentation licenses moving 
%from the GFDL to the CC-BY-SA (e.g., Wikipedia).
%Using the CC-BY-SA license maintains the book's 
%strong copyleft tradition while making it even more straightforward 
%for new authors to reuse this material as they see fit.

Eu sinto que este livro serve de exemplo sobre como materiais abertos
(gratuito) são importantes para o futuro da educação, e quero agradecer ao
Allen B. Downey e à editora da Universidade de Cambridge por sua decisão de
tornar este livro disponível sob uma licença aberta de direitos autorais. Eu
espero que eles fiquem satisfeitos com os resultados dos meus esforços e eu
desejo que você leitor esteja satisfeito com \emph{nosso} esforço coletivo.

%I feel that this book serves an example of why open 
%materials are so important to the future of education,
%and want to thank Allen B. Downey and Cambridge University
%Press for their forward-looking decision to make the book available
%under an open copyright.   I hope they are pleased with the 
%results of my efforts and I hope that you the reader are pleased with
%\emph{our} collective efforts.

Eu quero fazer um agradecimento ao Allen B. Downey e Lauren Cowles por sua
ajuda, paciência, e instrução em lidar com este trabalho e resolver os
problemas de direitos autorais que cercam este livro.

%I would like to thank Allen B. Downey and Lauren Cowles for their help,
%patience, and guidance in dealing with and resolving the copyright 
%issues around this book.

Charles Severance\\
www.dr-chuck.com\\
Ann Arbor, MI, USA\\
9 de Setembro de 2013

%Charles Severance\\
%www.dr-chuck.com\\
%Ann Arbor, MI, USA\\
%September 9, 2013

Charles Severance é um Professor Associado à Escola de Informação da
Universidade de Michigan.

%Charles Severance is a 
%Clinical Associate Professor 
%at the University of Michigan School of Information.


