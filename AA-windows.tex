  O conteúdo deste arquivo é
  Copyright (c) 2009-  Charles R. Severance, Todos os direitos reservados
% The contents of this file is 
% Copyright (c) 2009-  Charles R. Severance, All Righs Reserved

\chapter{Programando Python no Windows}
%\chapter{Python Programming on Windows}
%

Neste apêndice, nós demonstraremos os passos para você
conseguir rodar Python no Windows. Existem muitos jeitos
de se fazer e a idéia aqui é escolher um modo que simplifique
o processo.
%In this appendix, we walk through a series of steps
%so you can run Python on Windows. There are many different 
%approaches you can take, and this is just one
%approach to keep things simple.
%

Primeiro, você precisa instalar um editor de programas.
Você pode não querer usar o Notepad ou o editor Microsoft
Word para editar programas Python. Programas devem ser arquivos
texto simples, então você precisará de um bom editor de arquivos
texto.
%First, you need to install a programmer editor.  You
%do not want to use Notepad or Microsoft Word to edit
%Python programs.  Programs must be in "flat-text" files
%and so you need an editor that is good at
%editing text files.
%

Nosso editor recomendado para Windows é o NotePad++, que pode
ser instalado a partir daqui:
%Our recommended editor for Windows is NotePad++ which
%can be downloaded and installed from:
%

\url{https://notepad-plus-plus.org/}
%\url{https://notepad-plus-plus.org/}

Faça o download da versão mais recente do Python 2 a partir
 do site oficial \url{www.python.org}
%Then download a recent version of Python 2 from the
%\url{www.python.org} web site.

\url{https://www.python.org/downloads/}
%\url{https://www.python.org/downloads/}

Uma vez que você instalou o Python, você deve ter uma nova
pasta em seu computador, tal como {\tt C:{\textbackslash}Python27}.
%Once you have installed Python, you should have a new
%folder on your computer like {\tt C:{\textbackslash}Python27}.
%

Para criar um programa Python, execute o NotePad++ a partir do
seu menu iniciar e salve o arquivo com o sufixo ``.py''. Para
este exercício, coloque uma pasta na sua Área de Trabalho chamada
{\tt py4inf}. É melhor utilizar nomes de pasta curtos e não ter 
nenhum tipo de espaço, acento ou caractere especial, seja na pasta
ou no nome do arquivo.
%To create a Python program, run NotePad++ from the Start Menu
%and save the file with a suffix of ``.py''.  For this
%exercise, put a folder on your Desktop named 
%{\tt py4inf}.  It is best to keep your folder names short
%and not to have any spaces in your folder or file name.
%

Vamos fazer o nosso primeiro programa Python:
%Let's make our first Python program be:
%

\beforeverb
\begin{verbatim}
print 'Hello Chuck'
\end{verbatim}
\afterverb

%\beforeverb
%\begin{verbatim}
%print 'Hello Chuck'
%\end{verbatim}
%\afterverb

Com exceção que você deve trocar para o seu nome. Salve o arquivo
em: {\tt Desktop{\textbackslash}py4inf{\textbackslash}prog1.py}.
%Except that you should change it to be your name.  Save the file
%into {\tt Desktop{\textbackslash}py4inf{\textbackslash}prog1.py}.
%

Então abra a janela de linha de comando. Isto varia de acordo com
a versão do Windows que você utiliza:
%Then open a command-line window.  Different versions of Windows
%do this differently:
%

\begin{itemize}
\item Windows Vista e Windows 7: Pressione {\bf Iniciar}
e então na janela de pesquisa que se abre, digite a palavra
{\tt command} e pressione enter.
%\begin{itemize}
%\item Windows Vista and Windows 7: Press {\bf Start}
%and then in the command search window enter the word
%{\tt command} and press enter.
%

\item Windows XP: Pressione {\bf Iniciar}, e {\bf Executar}, e 
então digite {\tt cmd} na caixa de diálogo e pressione {\bf OK}.
\end{itemize}
%\item Windows XP: Press {\bf Start}, then {\bf Run}, and 
%then enter {\tt cmd} in the dialog box and press {\bf OK}.
%\end{itemize}

Você verá uma janela de texto com um prompt que te mostrará
em qual pasta você se encontra.
%You will find yourself in a text window with a prompt that
%tells you what folder you are currently ``in''.  

Windows Vista and Windows-7: {\tt C:{\textbackslash}Users{\textbackslash}csev}\\
Windows XP: {\tt C:{\textbackslash}Documents and Settings{\textbackslash}csev}
%Windows Vista and Windows-7: {\tt C:{\textbackslash}Users{\textbackslash}csev}\\
%Windows XP: {\tt C:{\textbackslash}Documents and Settings{\textbackslash}csev}

Este é o seu ``diretório do usuário''. Agora nós precisamos
caminhar para a pasta onde você salvou o seu programa Python
utilizando os seguintes comandos:
%This is your ``home directory''.  Now we need to move into 
%the folder where you have saved your Python program using
%the following commands:
%

\beforeverb
\begin{verbatim}
C:\Users\csev\> cd Desktop
C:\Users\csev\Desktop> cd py4inf
\end{verbatim}
\afterverb
%\beforeverb
%\begin{verbatim}
%C:\Users\csev\> cd Desktop
%C:\Users\csev\Desktop> cd py4inf
%\end{verbatim}
%\afterverb

Então digite
%Then type 
%

\beforeverb
\begin{verbatim}
C:\Users\csev\Desktop\py4inf> dir 
\end{verbatim}
\afterverb
%\beforeverb
%\begin{verbatim}
%C:\Users\csev\Desktop\py4inf> dir 
%\end{verbatim}
%\afterverb

para listar os seus arquivos. Você verá o {\tt prog1.py} quando
você digitar o comando {\tt dir}.
%to list your files.  You should see the {\tt prog1.py} when 
%you type the {\tt dir} command.
%

Para executar o seu programa, simplesmente digite o nome do seu
arquivo no prompt de comando e pressione enter.
%To run your program, simply type the name of your file at the 
%command prompt and press enter.

\beforeverb
\begin{verbatim}
C:\Users\csev\Desktop\py4inf> prog1.py
Hello Chuck
C:\Users\csev\Desktop\py4inf> 
\end{verbatim}
\afterverb
%\beforeverb
%\begin{verbatim}
%C:\Users\csev\Desktop\py4inf> prog1.py
%Hello Chuck
%C:\Users\csev\Desktop\py4inf> 
%\end{verbatim}
%\afterverb

Você pode editar o arquivo no NotePad++, salvar, e então voltar para
a linha de comando e executar o seu programa de novo apenas digitando
o nome do arquivo na linha de comando.
%You can edit the file in NotePad++, save it, and then switch back
%to the command line and execute the program again by typing
%the file name again at the command-line prompt.
%

Se você estiver confuso na janela de comando, apenas feche e abra
uma nova.
%If you get confused in the command-line window, just close it
%and open a new one.
%

Dica: Você pode pressionar a ``seta para cima'' na linha de comando
para rolar e executar o último comando executado anteriormente.
%Hint: You can also press the ``up arrow'' at the command line to 
%scroll back and run a previously entered command again.
%

Você também deve olhar nas preferências do NotePad++ e configurar
para expandir os caracteres tab para serem quatro espaços. Isto irá
te ajudar bastante e não enfrentar erros de identação.
%You should also look in the preferences for NotePad++ and set it 
%to expand tab characters to be four spaces.  This will save you lots
%of effort looking for indentation errors.
%

Você pode encontrar maiores informações sobre editar e executar
programas Python em \url{www.py4inf.com}.
%You can also find further information on editing and running 
%Python programs at \url{www.py4inf.com}.
