% The contents of this file is 
% Copyright (c) 2009-  Charles R. Severance, All Righs Reserved

%\chapter{Contributions}
%\section*{Contributor List for ``Python for Informatics''}
\chapter{Contribuições}
\section*{Lista de Contribuidores para o ``Python para Informáticos''}

%Bruce Shields for copy editing early drafts,
Bruce Shields por copiar as edições dos primeiros rascunhos
Sarah Hegge,
Steven Cherry,
Sarah Kathleen Barbarow,
Andrea Parker,
Radaphat Chongthammakun,
Megan Hixon,
Kirby Urner,
Sarah Kathleen Barbrow,
Katie Kujala,
Noah Botimer,
Emily Alinder,
Mark Thompson-Kular,
James Perry,
Eric Hofer,
Eytan Adar,
Peter Robinson,
Deborah J. Nelson,
Jonathan C. Anthony,
Eden Rassette,
Jeannette Schroeder,
Justin Feezell,
Chuanqi Li,
Gerald Gordinier,
Gavin Thomas Strassel,
Ryan Clement,
Alissa Talley,
Caitlin Holman,
Yong-Mi Kim,
Karen Stover,
Cherie Edmonds,
Maria Seiferle,
Romer Kristi D. Aranas (RK),
Grant Boyer,
Hedemarrie Dussan,

% CONTRIB

%\section*{Preface for ``Think Python''}
\section*{Prefácio de ``Think Python''}


%\subsection*{The strange history of ``Think Python''}
\subsection*{A estranha história de ``Think Python''}

(Allen B. Downey)

%In January 1999 I was preparing to teach an introductory programming
%class in Java.  I had taught it three times and I was getting
%frustrated.  The failure rate in the class was too high and, even for
%students who succeeded, the overall level of achievement was too low.

Em Janeiro de 1999 estava me preparando para dar aulas para uma turma de 
Introdução à Programação em Java. Tinha ensinado por três vezes e estava 
ficando frustrado. O nível de reprovação na matéria estava muito alto e, 
mesmo para estudantes que tinham sido aprovados, o nível de aproveitamento foi 
muito baixo.

%One of the problems I saw was the books.  
%They were too big, with too much unnecessary detail about Java, and
%not enough high-level guidance about how to program.  And they all
%suffered from the trap door effect: they would start out easy,
%proceed gradually, and then somewhere around Chapter 5 the bottom would
%fall out.  The students would get too much new material, too fast,
%and I would spend the rest of the semester picking up the pieces.

Um dos problemas que eu percebi, eram os livros. Eles eram muito grandes, com 
muitos detalhes desnecessários sobre Java, e orientação insuficiente sobre 
como programar. E todos sofriam do efeito alçapão: eles iniciavam fácil, 
continuavam gradualmente, e então em algum lugar em torno do Capítulo 5 o 
chão se desfazia. Os estudantes teriam novos assuntos, muito rápido, e eu 
perderia o resto do semestre juntando as peças.

%Two weeks before the first day of classes, I decided to write my
%own book.
Duas semanas antes do primeiro dia de aula, decidi escrever meu próprio livro.

%My goals were:
Meus objetivos eram:

\begin{itemize}

%\item Keep it short.  It is better for students to read 10 pages
%than not read 50 pages.
\item Mantê-lo curto. É melhor para os estudantes lerem 10 páginas do que 
	estudar 50 páginas.

%\item Be careful with vocabulary.  I tried to minimize the jargon
%and define each term at first use.

\item Ser cuidadoso com o vocabulário. Tentei minimizar os jargões e definir 
	os termos na primeira vez que for utilizar.

%\item Build gradually. To avoid trap doors, I took the most difficult
%topics and split them into a series of small steps. 

\item Evolução gradual. Para evitar o efeito alçapão, peguei os tópicos mais 
	difíceis e dividi em séries de pequenos passos.

%\item Focus on programming, not the programming language.  I included
%the minimum useful subset of Java and left out the rest.

\item Foco em programação, não na linguagem. Eu inclui um subconjunto mínimo 
	de Java e deixei o resto de fora.
\end{itemize}

%I needed a title, so on a whim I chose \emph{How to Think Like
%a Computer Scientist}.

Eu precisava de um título, e por um capricho eu escolhi \emph{Como Pensar como 
um Cientista da Computação}.

%My first version was rough, but it worked.  Students did the reading,
%and they understood enough that I could spend class time on the hard
%topics, the interesting topics and (most important) letting the
%students practice.

Minha primeira versão foi dura, mas funcionou. Os estudantes leram e 
entenderam o suficiente que eu pudesse dedicar as aulas nos tópicos 
difíceis, os tópicos interessantes e (mais importantes) deixando os 
estudantes praticarem.

%I released the book under the GNU Free Documentation License,
%which allows users to copy, modify, and distribute the book.

Eu liberei o livro sob a Licença GNU Free Documentation, que permite aos 
usuários copiar, modificar e redistribuir o livro.

\index{GNU Free Documentation License}
\index{Free Documentation License, GNU}

%What happened next is the cool part.  Jeff Elkner, a high school
%teacher in Virginia, adopted my book and translated it into
%Python.  He sent me a copy of his translation, and I had the
%unusual experience of learning Python by reading my own book.

O que aconteceu depois disso foi a parte mais legal. Jeff Elkner, um 
professor de escola de ensino médio na Virgínia, adotou meu livro e adaptou 
para Python. Ele me enviou uma cópia da sua adaptação, e então tive a 
experiência de aprender Python lendo meu próprio livro.

%Jeff and I revised the book, incorporated a case study by
%Chris Meyers, and in 2001 we released \emph{How to Think Like
%a Computer Scientist: Learning with Python}, also under
%the GNU Free Documentation License.
%As Green Tea Press, I published the book and started selling
%hard copies through Amazon.com and college book stores.
%Other books from Green Tea Press are available at
%\url{greenteapress.com}.

Eu e Jeff revisamos o livro, incorporando um caso de estudo do Chris Meyers, 
e em 2001 nós liberamos \emph{Como Pensar como um Cientista da Computação: 
Aprendendo com Python}, também sob a licença GNU Free Documentation.
Pela Green Tea Press, publiquei o livro e comecei a vender cópias físicas 
pela Amazon.com e na livraria da Faculdade. Outros livros da Green Tea Press 
estão disponíveis no endereço \url{greenteapress.com}.

%In 2003 I started teaching at Olin College and I got to teach
%Python for the first time.  The contrast with Java was striking.
%Students struggled less, learned more, worked on more interesting
%projects, and generally had a lot more fun.

Em 2003 eu comecei a lecionar na faculdade de Olin e comecei a ensinar Python 
pela primeira vez. O contraste com Java foi impressionante. Os estudantes 
lutavam menos e aprendiam mais, trabalhavam com mais interesse nos projetos,
e normalmente se divertiam mais.

%Over the last five years I have continued to develop the book,
%correcting errors, improving some of the examples and
%adding material, especially exercises.  In 2008 I started work
%on a major revision---at the same time, I was
%contacted by an editor at Cambridge University Press who
%was interested in publishing the next edition.  Good timing!

Pelos últimos cinco anos eu continuei a desenvolver o livro, corrigindo erros, 
melhorando alguns dos exemplos e adicionando materiais, especialmente 
exercícios. Em 2008, comecei a trabalhar em uma nova revisão, ao mesmo 
tempo eu entrei em contato com um editor da Editora da Universidade de 
Cambridge que se interessou em publicar a próxima edição. Ótima oportunidade!

%I hope you enjoy working with this book, and that it helps
%you learn to program and think, at least a little bit, like
%a computer scientist.

Eu espero que você aprecie trabalhar neste livro, e que ele ajude você a 
aprender a programar e pense, pelo menos um pouco, como um cientista da 
computação.

%\subsection*{Acknowledgements for ``Think Python''}
\subsection*{Reconhecimentos para ``Think Python''}

(Allen B. Downey)

%First and most importantly, I thank Jeff Elkner, who
%translated my Java book into Python, which got this project
%started and introduced me to what has turned out to be my
%favorite language.

Primeiramente e mais importante, eu gostaria de agradecer Jeff Elkner, 
que adaptou meu livro em Java para Python, que pegou este projeto e me 
introduziu no que se tornou a minha linguagem favorita.

%I also thank Chris Meyers, who contributed several sections
%to \emph{How to Think Like a Computer Scientist}.

Eu também quero agradecer Chris Meyers, que contribuiu para muitas seções 
para \emph{Como Pensar como um Cientista da Computação}.

%And I thank the Free Software Foundation for developing
%the GNU Free Documentation License, which helped make
%my collaboration with Jeff and Chris possible.

E eu gostaria de agradecer a Free Software Foundation por desenvolver a 
Licença GNU Free Documentation, que ajudou na minha colaboração entre Jeff 
e Chris possível.

\index{Licença GNU Free Documentation}
\index{Free Documentation License, GNU}

%I also thank the editors at Lulu who worked on
%\emph{How to Think Like a Computer Scientist}.

Gostaria de agradecer aos editores da Lulu que trabalharam no \emph{How to 
Think Like a Computer Scientist}.

%I thank all the students who worked with earlier
%versions of this book and all the contributors (listed
%in an Appendix) who sent in corrections and suggestions.

Agradeço a todos os estudantes que trabalharam nas primeiras versões deste 
livro e todos os contribuidores (listados no apêndice) que enviaram correções 
e sugestões.

%And I thank my wife, Lisa, for her work on this book, and Green
%Tea Press, and everything else, too.

E agradeço a minha esposa, Lisa, pelo seu trabalho neste livro, e a Green 
Tea Press, por todo o resto.

Allen B. Downey \\
Needham MA\\

%Allen Downey is an Associate Professor of Computer Science at 
%the Franklin W. Olin College of Engineering.

Allen Downey é professor associado do curso de Ciência da Computação na 
Faculdade de Engenharia Franklin W. Olin.

%\section*{Contributor List for ``Think Python''}
\section*{Lista de contribuidores para o ``Think Python''}

%\index{contributors}
\index{contribuidores}

(Allen B. Downey)

%More than 100 sharp-eyed and thoughtful readers have sent in
%suggestions and corrections over the past few years.  Their
%contributions, and enthusiasm for this project, have been a
%huge help.

Mais de 100 leitores atentos e dedicados tem enviado sugestões e correções 
nos últimos anos. Suas contribuições e entusiasmo por este projeto, foram de 
grande ajuda.

%For the detail on the nature of each of the contributions from
%these individuals, see the ``Think Python'' text.

Para detalhes sobre a natureza das contribuições de cada uma destas pessoas, 
veja o texto the ``Think Python''.

Lloyd Hugh Allen,
Yvon Boulianne,
Fred Bremmer,
Jonah Cohen,
Michael Conlon,
Benoit Girard,
Courtney Gleason e Katherine Smith,
Lee Harr,
James Kaylin,
David Kershaw,
Eddie Lam,
Man-Yong Lee,
David Mayo,
Chris McAloon,
Matthew J. Moelter,
Simon Dicon Montford,
John Ouzts,
Kevin Parks,
David Pool,
Michael Schmitt,
Robin Shaw,
Paul Sleigh,
Craig T. Snydal,
Ian Thomas,
Keith Verheyden,
Peter Winstanley,
Chris Wrobel,
Moshe Zadka,
Christoph Zwerschke,
James Mayer,
Hayden McAfee,
Angel Arnal,
Tauhidul Hoque e Lex Berezhny,
Dr. Michele Alzetta,
Andy Mitchell,
Kalin Harvey,
Christopher P. Smith,
David Hutchins,
Gregor Lingl,
Julie Peters,
Florin Oprina,
D.~J.~Webre,
Ken,
Ivo Wever,
Curtis Yanko,
Ben Logan,
Jason Armstrong,
Louis Cordier,
Brian Cain,
Rob Black,
Jean-Philippe Rey da Ecole Centrale Paris,
Jason Mader da George Washington University fez uma série
Jan Gundtofte-Bruun,
Abel David e Alexis Dinno,
Charles Thayer,
Roger Sperberg,
Sam Bull,
Andrew Cheung,
C. Corey Capel,
Alessandra,
Wim Champagne,
Douglas Wright,
Jared Spindor,
Lin Peiheng,
Ray Hagtvedt,
Torsten H\"{u}bsch,
Inga Petuhhov,
Arne Babenhauserheide,
Mark E. Casida,
Scott Tyler,
Gordon Shephard,
Andrew Turner,
Adam Hobart,
Daryl Hammond e Sarah Zimmerman,
George Sass,
Brian Bingham,
Leah Engelbert-Fenton,
Joe Funke,
Chao-chao Chen,
Jeff Paine,
Lubos Pintes,
Gregg Lind e Abigail Heithoff,
Max Hailperin,
Chotipat Pornavalai,
Stanislaw Antol,
Eric Pashman,
Miguel Azevedo,
Jianhua Liu,
Nick King,
Martin Zuther,
Adam Zimmerman,
Ratnakar Tiwari,
Anurag Goel,
Kelli Kratzer,
Mark Griffiths,
Roydan Ongie,
Patryk Wolowiec,
Mark Chonofsky,
Russell Coleman,
Wei Huang,
Karen Barber,
Nam Nguyen,
St\'{e}phane Morin,
Fernando Tardio,
%and
e
Paul Stoop.

