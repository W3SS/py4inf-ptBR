% LaTeX source for ``Python for Informatics: Exploring Information''
% Copyright (c)  2010-  Charles R. Severance, All Rights Reserved

\chapter{Variveis, expresses e Declaraes}
%\chapter{Variables, expressions, and statements}
\section{Valores e tipos}
%\section{Values and types}
\index{value}
\index{type}
\index{string}

%A {\bf value} is one of the basic things a program works with,
%like a letter or a
%number.  The values we have seen so far
%are {\tt 1}, {\tt 2}, and
%\verb"'Hello, World!'"
Um {\bf valor}  uma das coisas bsicas com a qual um programa trabalha, como uma letra ou um nmero.
Os valores que vimos at agora so
so {\tt 1}, {\tt 2}, and
\verb"'Ol, Mundo!'"

%These values belong to different {\bf types}:
%{\tt 2} is an integer, and \verb"'Hello, World!'" is a {\bf string},
%so called because it contains a ``string'' of letters.
%You (and the interpreter) can identify
%strings because they are enclosed in quotation marks.
Estes valores pertencem a digerentes {\bf tipos}:
{\tt 2}  um inteiro, e \verb"'ol, Mundo!'"  uma {\bf string},
Assim chamada por conter uma ``cadeia'' de letras.
Voc (e o interpretador) podem identificar strings 
porque elas aparecem entre aspas.

\index{quotation mark}

%The {\tt print} statement also works for integers.  We use the 
%{\tt python} command to start the interpreter.
A declarao {\tt print} tambm funciona com inteiros. Ns usamos o 
comando %{\tt python} para iniciar o interpretador.

\beforeverb
\begin{verbatim}
python
>>> print 4
4
\end{verbatim}
\afterverb
%
%If you are not sure what type a value has, the interpreter can tell you.
Se voc no tem certeza que tipo tem um valor, o interpretador pode te dizer.

\beforeverb
\begin{verbatim}
>>> type('Hello, World!')
<type 'str'>
>>> type(17)
<type 'int'>
\end{verbatim}
\afterverb
% 
%Not surprisingly, strings belong to the type {\tt str} and
%integers belong to the type {\tt int}.  Less obviously, numbers
%with a decimal point belong to a type called {\tt float},
%because these numbers are represented in a
%format called {\bf floating point}.
No surpreendentemente, strings pertencem ao tipo {\tt str} e 
inteiros pertencem ao tipo {\tt int}. Menos, obviamente, nmeros 
com ponto decimal que pertencem a um tipo chamado {\tt float},
uma vez que estes nmeros so representados em um formato 
chamado {\bf ponto flutuante}.

\index{type}
\index{string type}
\index{type!str}
\index{int type}
\index{type!int}
\index{float type}
\index{type!float}

\beforeverb
\begin{verbatim}
>>> type(3.2)
<type 'float'>
\end{verbatim}
\afterverb
%
%What about values like \verb"'17'" and \verb"'3.2'"?
%They look like numbers, but they are in quotation marks like
%strings.
E quanto a valores como \verb"'17'" e \verb"'3.2'"?
Eles se parecem com nmeros, mas eles so, quando entre aspas, 
strings.

\index{quotation mark}

\beforeverb
\begin{verbatim}
>>> type('17')
<type 'str'>
>>> type('3.2')
<type 'str'>
\end{verbatim}
\afterverb
%
%They're strings.
Eles so strings.

%When you type a large integer, you might be tempted to use commas
%between groups of three digits, as in {\tt 1,000,000}.  This is not a
%legal integer in Python, but it is legal:
Quando voc digita um integer [nmero inteiro  N do T] grande, voc pode ficar tentado a utilizar vrgulas 
entre os grupos de trs dgitos, como em {\tt 1,000,000}. Este no  um 
nmero vlido em Python, no entanto ele  vlido:

\beforeverb
\begin{verbatim}
>>> print 1,000,000
1 0 0
\end{verbatim}
\afterverb
%
%Well, that's not what we expected at all!  Python interprets {\tt
%  1,000,000} as a comma-separated sequence of integers, which it
%prints with spaces between.

Bem, de toda forma, isto no  o que ns espervamos!  Python interpreta {\tt
  1,000,000} como  uma sequncia de integers separados por vrgulas, o qual 
imprimi com espaos entre eles.

\index{semantic error}
\index{error!semantic}
\index{error message}

%This is the first example we have seen of a semantic error: the code
%runs without producing an error message, but it doesn't do the
%``right'' thing.

Este  o primeiro exemplo que vemos de um erro semntico: o cdigo 
executa sem produzir uma mensagem de erro, mas ele no faz a 
coisa ``certa''.

%\section{Variables}
\section{Variveis}
\index{variable}
\index{assignment statement}
\index{statement!assignment}

%One of the most powerful features of a programming language is the
%ability to manipulate {\bf variables}.  A variable is a name that
%refers to a value.

Uma das mais poderosas caractersticas de uma linguagem de programao  a 
capacidade de manipular {\bf variveis). Uma varivel  um nome 
que se refere a um valor.

%An {\bf assignment statement} creates new variables and gives
%them values:
Um {\bf comando de atribuio) cria novas variveis e d 
valores a elas:

\beforeverb
\begin{verbatim}
>>> message = 'And now for something completely different'
>>> n = 17
>>> pi = 3.1415926535897931
\end{verbatim}
\afterverb
%
%This example makes three assignments.  The first assigns a string
%to a new variable named {\tt message};
%the second assigns the integer {\tt 17} to {\tt n}; the third
%assigns the (approximate) value of $\pi$ to {\tt pi}.
Este exemplo faz trs atribuies. O primeiro atribui uma string 
a uma nova varivel chamada {\tt message}; 
o segundo atribui o integer {\tt 17)  varivel {\tt n); o terceiro 
atribui valor (aproximado) de $\pi$  varivel {\tt pi}.
%To display the value of a variable, you can use a print statement:
Para mostrar o valor de uma varivel, voc pode usar o comando print.
\beforeverb
\begin{verbatim}
>>> print n
17
>>> print pi
3.14159265359
\end{verbatim}
\afterverb
%
%The type of a variable is the type of the value it refers to.
O tipo de uma varivel  o tipo do valor ao qual ela se refere.
\beforeverb
\begin{verbatim}
>>> type(message)
<type 'str'>
>>> type(n)
<type 'int'>
>>> type(pi)
<type 'float'>
\end{verbatim}
\afterverb
%

%\section{Variable names and keywords}
\section{Nomes de variveis e palvras reservadas}
\index{keyword}

%Programmers generally choose names for their variables that
%are meaningful and document what the variable is used for.
Programadores geralmente escolhem nomes que tenham algum significado, para 
suas variveis.Assim eles documentam para que fim a varivel ser utilizada.

%Variable names can be arbitrarily long.  They can contain
%both letters and numbers, but they cannot start with a number.
%It is legal to use uppercase letters, but it is a good idea
%to begin variable names with a lowercase letter (you'll
%see why later).
Nomes de variveis podem ser arbitrariamente longos. Eles podem conter 
tanto letras, quanto nmeros, porm eles no podem comear com um nmero.
 vlido usar letras maisculas, porm  uma boa prtica comear o nome de uma varivel 
com uma letra minscula (voc ver o porqu, mais tarde).

%The underscore character (\verb"_") can appear in a name.
%It is often used in names with multiple words, such as
\verb"my_name" or \verb"airspeed_of_unladen_swallow".
O caractere sublinhado (\verb"_") pode aparecer no nome.
Ele  frequentemente usado em nomes com mltiplas palavras, como 
\verb"my_name" ou \verb"airspeed_of_unladen_swallow".

%Variable names can start with an underscore character, but
%we generally avoid doing this unless we are writing library 
%code for others to use.
Nomes de variveis podem comear como caracter sublinhado, mas
ns, geralmente, evitamos isto, a menos que estejamos escrevendo uma biblioteca
de cdigo para outros usarem.
\index{underscore character}

If you give a variable an illegal name, you get a syntax error:
Se voc der a uma varivel um nome invlido, voc receber um erro de sintaxe.
\beforeverb
\begin{verbatim}
>>> 76trombones = 'big parade'
SyntaxError: invalid syntax
>>> more@ = 1000000
SyntaxError: invalid syntax
>>> class = 'Advanced Theoretical Zymurgy'
SyntaxError: invalid syntax
\end{verbatim}
\afterverb
%
%{\tt 76trombones} is illegal because it begins with a number.
%{\tt more@} is illegal because it contains an illegal character, {\tt
%@}.  But what's wrong with {\tt class}?
%It turns out that {\tt class} is one of Python's {\bf keywords}.  The
%interpreter uses keywords to recognize the structure of the program,
%and they cannot be used as variable names.
{\tt 76trombones)  invlida porqu ela comea com um nmero.
{\tt more@)  invlida porqu ela contm um caractere invlido,{\tt 
@). Mas o qu h de errado com {\tt class)?
Acontece que a palavra {\tt class)  uma Palavra Reservada do Python {\bf keywords}.
O interpretador usa as Palavras Reservadas para reconhecer a estrutura do programa,
e elas no podem ser usadas como nomes de variveis.
\index{keyword}

%Python reserves 31 keywords\footnote{In Python 3.0, {\tt exec} is no
%longer a keyword, but {\tt nonlocal} is.} for its use:
Python reserva 31 Palavras Reservadas \footnote{In Python 3.0, {\tt exec} no  
mais uma patalvra reservada, mas {\tt nonlocal} .} para seu uso:

\beforeverb
\begin{verbatim}
and       del       from      not       while    
as        elif      global    or        with     
assert    else      if        pass      yield    
break     except    import    print              
class     exec      in        raise              
continue  finally   is        return             
def       for       lambda    try
\end{verbatim}
\afterverb
%
%You might want to keep this list handy.  If the interpreter complains
%about one of your variable names and you don't know why, see if it
%is on this list.
Voc pode querer manter esta lista ao alcance das mos. Se o interpretador reclamar
sobre um de seus nomes de varivel e voc no souber o porqu, verifique se ela 
se encontra na lista.

%\section{Statements}
\section{Declaraes}

%A {\bf statement} is a unit of code that the Python interpreter can
%execute.  We have seen two kinds of statements: print
%and assignment.
Uma {\bf declarao)  uma unidade de cdigo a qual o interpretador Python 
pode executar. Ns temos visto dois tipos de declaraes: impresso 
e atribuio.

\index{statement}
\index{interactive mode}
\index{script mode}

%When you type a statement in interactive mode, the interpreter
%executes it and displays the result, if there is one.
Quando voc digita uma declarao no modo interativo, o interpretador 
a executa e mostra o resultado, se houver um.

%A script usually contains a sequence of statements.  If there
%is more than one statement, the results appear one at a time
%as the statements execute.
Um script usualmente contm uma sequncia de declaraes. Se houver 
mais de uma declarao, os resultados aparecem um de cada vez 
conforme as declaraes so executadas.

%For example, the script
Por exemplo, o script

\beforeverb
\begin{verbatim}
print 1
x = 2
print x
\end{verbatim}
\afterverb
%
%produces the output
Produz a sada

\beforeverb
\begin{verbatim}
1
2
\end{verbatim}
\afterverb
%
%The assignment statement produces no output.
A declarao de atribuio no produz sada.

%\section{Operators and operands}
\section{Operadores e operandos}
\index{operator, arithmetic}
\index{arithmetic operator}
\index{operand}
\index{expression}

%{\bf Operators} are special symbols that represent computations like
%addition and multiplication.  The values the operator is applied to
%are called {\bf operands}.
{\bf Operadores) so smbolos especiais que representam clculos como 
adio e multiplicao. Os valores aos quais os operadores so aplicados 
so chamados de operandos {\bf operands}.

%The operators {\tt +}, {\tt -}, {\tt *}, {\tt /}, and {\tt **}
%perform addition, subtraction, multiplication, division, and
%exponentiation, as in the following examples:
Os operadores {\tt +), {\tt -), {\tt *), {\tt /), e {\tt **) 
realizam, adio, subtrao, mumltiplicao, diviso 
e exponenciao, como no exemplo a seguir:

\beforeverb
\begin{verbatim}
20+32   hour-1   hour*60+minute   minute/60   5**2   (5+9)*(15-7)
\end{verbatim}
\afterverb
%
%The division operator might not do what you expect:
O operador de diviso pode no fazer o que voc espera: 

\beforeverb
\begin{verbatim}
>>> minute = 59
>>> minute/60
0
\end{verbatim}
\afterverb
%
%The value of {\tt minute} is 59, and in conventional arithmetic 59
%divided by 60 is 0.98333, not 0.  The reason for the discrepancy is
%that Python is performing {\bf floor division}\footnote{In Python 3.0,
%the result of this division is a {\tt float}.  
%In Python 3.0, the new operator
%{\tt //} performs integer division.}.
O valor de {\tt minute)  59, e na aritimtica convencional 59 
dividido por 60  0.98333, no 0. A razo para esta discrepncia  
pelo fato de que o Python realiza um {\bf floor division}\footnote{In Python 3.0,
%o resuldado desta diviso  do tipo {\tt float}.  
%In Python 3.0, o novo operador 
%{\tt //} realiza uma diviso to tipo integer.}

\index{Python 3.0}
\index{floor division}
\index{floating-point division}
\index{division!floor}
\index{division!floating-point}

%When both of the operands are integers, the result is also an
%integer; floor division chops off the fractional
%part, so in this example it truncates the answer to zero.
Quando os dois operandos so integers, o resultado , tambm,
um integer; floor division corta a parte fracionria,
portanto, neste exemplo o resultado foi arredondado para zero.

%If either of the operands is a floating-point number, Python performs
%floating-point division, and the result is a {\tt float}:
Se um dos operandos  um nmero do tipo ponto flutuante, Python realiza 
uma diviso de ponto flutuante, e o resultado  um {\tt float}:

\beforeverb
\begin{verbatim}
>>> minute/60.0
0.98333333333333328
\end{verbatim}
\afterverb


\section{Expressions}
\section{Expresses}

%An {\bf expression} is a combination of values, variables, and operators.
%A value all by itself is considered an expression, and so is
%a variable, so the following are all legal expressions
%(assuming that the variable {\tt x} has been assigned a value):
Uma {\bf expresso)  uma combinao de valores, variveis e operadores. 
Um valor, por si s,  considerado uma expresso, e portanto,
uma varivel, ento o que segue so todas expresses vlidas 
(assumindo que a varivel {\tt x) tenha recebido um valor):
\index{expression}
\index{evaluate}

\beforeverb
\begin{verbatim}
17
x
x + 17
\end{verbatim}
\afterverb
%
%If you type an expression in interactive mode, the interpreter
%{\bf evaluates} it and displays the result:
Se voc digita uma expresso no modo interativo, o interpretador
a {\bf calcula} e mostra o resultado:
\beforeverb
\begin{verbatim}
>>> 1 + 1
2
\end{verbatim}
\afterverb
%
%But in a script, an expression all by itself doesn't
%do anything!  This is a common
%source of confusion for beginners.
Mas em um script, uma expresso por si s no 
faz nada! Isto  uma fonte comum 
de confuso para iniciantes.

%5                                                                                                         \begin{ex}
%Type the following statements in the Python interpreter to see
%what they do:
Exerccio 2.1 Digite a seguinte declaraaono interpretador do Python para ver 
o que ele faz:
\beforeverb
\begin{verbatim}
5
x = 5
x + 1
\end{verbatim}
\afterverb
%
\end{ex}


\section{Order of operations}
\section{Ordem das operaes}
\index{order of operations}
\index{rules of precedence}
\index{PEMDAS}

%When more than one operator appears in an expression, the order of
%evaluation depends on the {\bf rules of precedence}.  For
%mathematical operators, Python follows mathematical convention.
%The acronym {\bf PEMDAS} is a useful way to
%remember the rules:
Quando mais de um operador aparece em uma expresso, a ordem de 
avaliao depende das {\bf regras de precedncia). Para 
operadores matemticos. Python segue a conveno matemtica.
O Acrnimo {\bf PEMDAS)  uma modo til para lembrar as regras:
\index{parentheses!overriding precedence}

\begin{itemize}

%\item {\bf P}arentheses have the highest precedence and can be used 
%to force an expression to evaluate in the order you want. Since
%expressions in parentheses are evaluated first, {\tt 2 * (3-1)} is 4,
%and {\tt (1+1)**(5-2)} is 8. You can also use parentheses to make an
%expression easier to read, as in {\tt (minute * 100) / 60}, even
%if it doesn't change the result.
%\item {\bf P}arenteses tm a mais alta precedncia e pode ser usado 
para forar que uma expresso seja calculada na ordem que voc deseja. Como as 
expresses entre parnteses so avalidas primeiro, {\tt 2 * (3-1))  4, 
e {\tt (1+1)**(5-2))  8. Voc tambm pode usar parnteses para tornar uma 
expresso mais fcil de ser lida, como em {\tt (minute * 100) / 60), mesmo 
que isto no mude o resultado.

%\item {\bf E}xponentiation has the next highest precedence, so
%{\tt 2**1+1} is 3, not 4, and {\tt 3*1**3} is 3, not 27.
\item {\bf E}xponenciao  a prxima precedncia mais alta,
{\tt ento 2**1+1)  3, no 4, e %{\tt 3*1**3)  3, no 27.

%\item {\bf M}ultiplication and {\bf D}ivision have the same precedence,
%which is higher than {\bf A}ddition and {\bf S}ubtraction, which also
%have the same precedence.  So {\tt 2*3-1} is 5, not 4, and
%{\tt 6+4/2} is 8, not 5.
\item {\bf M}ultiplicao e Diviso tm a mesma precedncia,
a qual  mais alta que {\bf A}dio e {\bf S)ubtrao, que tambm 
tm a mesma precedncia entre si. Ento {\tt 2*3-1)  5, no 4, e 
{\tt6+4/2)  8, no 5.


%\item Operators with the same precedence are evaluated from left to 
%right.  So the expression {\tt 5-3-1} is 1, not 3, because the
%{\tt 5-3} happens first and then {\tt 1} is subtracted from {\tt 2}.
\itemOperadores com a mesma precedncia so calculados da esquerda para 
direita. Portanto na expresso {\tt 5-3-1)  1, no 3 pois o 
{\tt 5-3) acontence primeiro e ento o {\tt 1)  subtrado de {\tt 2).
\end{itemize}

%When in doubt, always put parentheses in your expressions to make sure
%the computations are performed in the order you intend.
Quando em dvida, sempres utilize parnteses em suas expresses para ter certeza 
de que os clculos sero realizados na ordem que voc deseja.

\section{Modulus operator}
\section{O operador Mdulo}

\index{modulus operator}
\index{operator!modulus}

%The {\bf modulus operator} works on integers and yields the remainder
%when the first operand is divided by the second.  In Python, the
%modulus operator is a percent sign (\verb"%").  The syntax is the same
%as for other operators:
O operador {\bf mdulo) funciona em integers e fornece o resto,
quando o primeiro operador  dividido pelo segundo. No Python, o 
operador mdulo  um sinal de percentual (\verb"%"). A sintaxe  a mesma 
dos outros operadores:
\beforeverb
\begin{verbatim}
>>> quotient = 7 / 3
>>> print quotient
2
>>> remainder = 7 % 3
>>> print remainder
1
\end{verbatim}
\afterverb
%
%So 7 divided by 3 is 2 with 1 left over.
Portanto, 7 dividido por 3  igual a 2, com resto 1.

%The modulus operator turns out to be surprisingly useful.  For
%example, you can check whether one number is divisible by another---if
%{\tt x \% y} is zero, then {\tt x} is divisible by {\tt y}.
O operador mdulo apresenta-se surpreendentemente til. Por 
exemplo, voc pode checar se um nnero  divisvel por outro -- se 
{\tt x \% y}  zero, ento {\tt x}  divivisvel por {\tt y}.

\index{divisibility}

%You can also extract the right-most digit
%or digits from a number.  For example, {\tt x \% 10} yields the
%right-most digit of {\tt x} (in base 10).  Similarly, {\tt x \% 100}
%yields the last two digits.
Voc pod, tanbm, extrair os dgitos mais  direita 
de um nmero. Por exemplo, {\tt x \% 10} fornece o 
dgito mais  direita de {\tt x} (na base 10). Similarmente, {\tt x \% 100}
fornece os ltimos dois dgitos.


%\section{String operations}
\section{Operaes com Strings}
\index{string!operation}
\index{operator!string}

%The {\tt +} operator works with strings, but it
%is not addition in the mathematical sense. Instead it performs
%{\bf concatenation}, which means joining the strings by
%linking them end to end.  For example:
O operador {\tt +} funciona com strings, mas ele 
no  uma adio no sentido matemtico. Ao invs disto, ele realiza 
{\bfconcatenao), que significa juntar as strings,
vinculando-as de ponta-a-ponta. Por exemplo:
\index{concatenation}

\beforeverb
\begin{verbatim}
>>> first = 10
>>> second = 15
>>> print first+second
25
>>> first = '100'
>>> second = '150'
>>> print first + second
100150
\end{verbatim}
\afterverb
%
%The output of this program is {\tt 100150}.
A sada deste programa  {\tt 100150}.

%\section{Asking the user for input}
\section{Solicitando dados de entrada para o usurio}
\index{keyboard input}

%Sometimes we would like to take the value for a variable from the user
%via their keyboard.
%Python provides a built-in function called \verb"raw_input" that gets
%input from the keyboard\footnote{In Python 3.0, this function is named
%  {\tt input}.}.  When this function is called, the program stops and
%waits for the user to type something.  When the user presses {\sf
%  Return} or {\sf Enter}, the program resumes and \verb"raw_input"
%returns what the user typed as a string.
Algumas vezes gostaramos de solicitar, do usurio, o valor para uma varivel
por meio do teclado.
Python fornece uma funo interna chamada \verb"raw_input" que recebe
dados de entrada a partir do teclado\footnote{In Python 3.0, esta funo  chamda de 
{\tt input}.}. Quando esta funo  chamada, o program para e 
espera para que o usurio digite algo. Quando o usurio pressiona o 
{\sf Return) ou {\sf Enter), o programa continua e a funo \verb"raw_input"
retorna o que o usurio digitou, como uma string.

\index{Python 3.0}
\index{raw\_input function}
\index{function!raw\_input}

\beforeverb
\begin{verbatim}
>>> input = raw_input()
Some silly stuff
>>> print input
Some silly stuff
\end{verbatim}
\afterverb
%
%Before getting input from the user, it is a good idea to print a
%prompt telling the user what to input.  You can pass a string
%to \verb"raw_input" to be displayed to the user before pausing
%for input:
Antes de receber os dados de entrada vindos do usurio,  uma boa idia imprimir uma 
mensagem, dizendo ao usurio que dado deve ser informado. Voc pode passar uma string 
para a funo \verb"raw_input" para ser mostrada para o usurio antes da parada 
para a entrada de dados:

\index{prompt}

\beforeverb
\begin{verbatim}
>>> name = raw_input('What is your name?\n')
What is your name?
Chuck
>>> print name
Chuck
\end{verbatim}
\afterverb
%
%The sequence \verb"\n" at the end of the prompt represents a {\bf newline},
%which is a special character that causes a line break.
%That's why the user's input appears below the prompt.
A sequncia \verb"\n" no final da mensagem representa uma {\bf nova linha},
que  um caractere especial que provoca a quebra de linha. 
 por este motivo que os dados de entrada informados pelo usurio aparecem abaixo da mensagem.

\index{newline}

%If you expect the user to type an integer, you can try to convert
%the return value to {\tt int} using the {\tt int()} function:
Se voc espera que o usurio digite um integer, voc pode tentar converter 
o valor retornado para {\tt int} usando a funo {\tt int()}: 

\beforeverb
\begin{verbatim}
>>> prompt = 'What...is the airspeed velocity of an unladen swallow?\n'
>>> speed = raw_input(prompt)
What...is the airspeed velocity of an unladen swallow?
17
>>> int(speed)
17
>>> int(speed) + 5
22
\end{verbatim}
\afterverb
%
%But if the user types something other than a string of digits,
%you get an error:
Porm, se o usurio digita algo diferente de um conjunto de nmeros, 
voc recebe um erro:
\beforeverb
\begin{verbatim}
>>> speed = raw_input(prompt)
What...is the airspeed velocity of an unladen swallow?
What do you mean, an African or a European swallow?
>>> int(speed)
ValueError: invalid literal for int()
\end{verbatim}
\afterverb
%
%We will see how to handle this kind of error later.
Ns veremos como tratar este tipo de erros mais tarde.

\index{ValueError}
\index{exception!ValueError}


%\section{Comments}
\section{Comentrios}
\index{comment}

%As programs get bigger and more complicated, they get more difficult
%to read.  Formal languages are dense, and it is often difficult to
%look at a piece of code and figure out what it is doing, or why.
Como os programas ficam maiores e mais complicado, eles ficam mais difceis 
de ler. Linguagens formais so densas, e muitas vezes  difcil 
olhar para um pedao de cdigo e descobrir o que ele est fazendo, ou porqu.

%For this reason, it is a good idea to add notes to your programs to explain
%in natural language what the program is doing.  These notes are called
%{\bf comments}, and in Python they start with the \verb"#" symbol:
Por esta razo,  uma boa idia adicionar notas em seu programa para explicar,
em linguagem natural, o que o program est fazendo. Estas notas so chamadas 
de {\bf comentrios), e, em Python, elas comeam com o smbolo \verb"#":

\beforeverb
\begin{verbatim}
# compute the percentage of the hour that has elapsed
percentage = (minute * 100) / 60
\end{verbatim}
\afterverb
%
%In this case, the comment appears on a line by itself.  You can also put
%comments at the end of a line:
Neste caso, o comentrio aparece sozinho em uma linha. Voc pode, tambm, coloar
os comentrio no final da linha:

\beforeverb
\begin{verbatim}
percentage = (minute * 100) / 60     # percentage of an hour
\end{verbatim}
\afterverb
%
%Everything from the {\tt \#} to the end of the line is ignored---it
%has no effect on the program.
%
%Comments are most useful when they document non-obvious features of
%the code.  It is reasonable to assume that the reader can figure out
%\emph{what} the code does; it is much more useful to explain \emph{why}.
Todos os caracteres a partir do {\tt \#), at o fim da linha so ignorados--eles 
no tm efeito sobre o programa.
%
Comentrios so mais teis quando no documentam caractersticas no obvias 
do cdigo. razovel assumir que o leitor pode descobrir 
\emph{o que) o cdigo faz;  muito mais til explicar o \emph{porqu).  
This comment is redundant with the code and useless:
Este comentrio  redundante e intil dentro do cdigo:
\beforeverb
\begin{verbatim}
v = 5     # assign 5 to v
\end{verbatim}
\afterverb
%
%This comment contains useful information that is not in the code:
Este comentrio contem informaes teis que no esto no cdigo.
\beforeverb
\begin{verbatim}
v = 5     # velocity in meters/second. 
\end{verbatim}
\afterverb
%
%Good variable names can reduce the need for comments, but
%long names can make complex expressions hard to read, so there is
%a trade-off.
Bons nomes de variveis podem reduzir a necessidade de comentrios, porm, 
nomes longos podem tornar expresses complexas difceis de serem lidas, ento devemos
balancear. 
 
%\section{Choosing mnemonic variable names}
\section{Escolhendo nomes de variveis mnemnicos}

\index{mnemonic}

%As long as you follow the simple rules of variable naming, and avoid
%reserved words, you have a lot of choice when you name your variables.
%In the beginning, this choice can be confusing both when you read a 
%program and when you write your own programs.  For example, the
%following three programs are identical in terms of what they accomplish,
%but very different when you read them and try to understand them.
Contanto que voc siga as regras simples de nomenclatura de variveis, e evite 
Palavras Reservadas, voc tem muitas escolhas quando voc nomeia suas variveis. 
No incio, esta escolha pode ser confusa, tanto quando voc l um 
programa, quanto quando voc escreve seus prprios programas. Por exemplo, os 
trs programas a seguir so idnticos em termos do que realizam, 
mas muito diferente quando voc os l e tenta compreend-los.
\beforeverb
\begin{verbatim}
a = 35.0
b = 12.50
c = a * b
print c

hours = 35.0
rate = 12.50
pay = hours * rate
print pay

x1q3z9ahd = 35.0
x1q3z9afd = 12.50
x1q3p9afd = x1q3z9ahd * x1q3z9afd
print x1q3p9afd
\end{verbatim}
\afterverb
%
%The Python interpreter sees all three of these programs as \emph{exactly the 
%same} but humans see and understand these programs quite differently.  
%Humans will most quickly understand the {\bf intent} 
%of the second program because the 
%programmer has chosen variable names that reflect their intent
%regarding what data will be stored in each variable.
O interpretador Python v todos os trs programas \emph{exatamente como o 
mesmo), mas os seres humanos veem e entendem esses programas de forma bastante diferente.
Os seres humanos entendero mais rapidamente a {\bf inteno)
do segundo programa, porque o 
programador escolheu nomes de variveis que refletem a sua inteno
sobre os dados que sero armazenados em cada varivel.
%We call these wisely chosen variable names ``mnemonic variable names''.  The
%word \emph{mnemonic}\footnote{See 
%\url{http://en.wikipedia.org/wiki/Mnemonic}
%for an extended description of the word ``mnemonic''.} 
%means ``memory aid''.
%We choose mnemonic variable names to help us remember why we created the variable
%in the first place.
Ns chamamos esses nomes de variveis sabiamente escolhidos de ``nomes de variveis mnemnicos''. A 
palavra \emph{mnemnico)\footnote{veja 
\url{http://en.wikipedia.org/wiki/Mnemonic}
para uma descrio completad da palavra ``mnemnico''.} 
significa ``auxiliar de memria''. 
Ns escolhemos os nomes de variveis mnemnicos para nos ajudar a lembrar o motivo pelo qual criamos a varivel, 
em primeiro lugar.

%While this all sounds great, and it is a very good idea to use mnemonic variable
%names, mnemonic variable names can get in the way of a beginning programmer's 
%ability to parse and understand code.  This is because beginning programmers 
%have not yet memorized the reserved words (there are only 31 of them) and sometimes
%variables with names that are too descriptive start to look like 
%part of the language and not just well-chosen variable names.
Isso tudo soa muito bem, e  uma boa idia usar nomes de varivel mnemnicos, 
nomes de variveis mnemnicos podem atrapalhar a capacidade de anlise e 
entendimento do cdigo de um programador iniciante. Isto acontece porque os programadores iniciantes 
ainda no memorizaram as palavras reservadas (existem apenas 31 delas) e, por vezes, 
variveis que tm nomes muito descritivos podem parecer 
parte da linguagem e no apenas nomes de variveis bem escolhidas.
%Take a quick look at the following Python sample code which loops through some data. 
%We will cover loops soon, but for now try to just puzzle through what this means:   
D uma olhada rpida no seguinte exemplo de cdigo Python que percorre alguns dados. 
Ns vamos falar sobre loops em breve, mas por agora apenas tente se confundir com o que isto significa:

\beforeverb
\begin{verbatim}
for word in words:
    print word
\end{verbatim}
\afterverb
%
%What is happening here?  Which of the tokens (for, word, in, etc.) are reserved words
%and which are just variable names?  Does Python understand at a fundamental level 
%the notion of words?  Beginning programmers have 
%trouble separating what parts of the
%code \emph{must} be the same as this example and what parts of the code are simply
%choices made by the programmer.
%
%The following code is equivalent to the above code:
O que esta acontecendo aqui? Qual das palavras (for, word, in, etc.) so palavras reservadas 
e quais so apenas nomes de variveis? O Python entende em um nvel fundamental 
a noo de palavras? Programadores iniciantes tm 
dificuldade para separar quais partes 
do cdigo \emph{devem) ser o mesmo que este exemplo e que partes do cdigo so simplesmente 
as escolhas feitas pelo programador.
%
O cdigo a seguir  equivalente ao cdigo acima:

\beforeverb
\begin{verbatim}
for slice in pizza:
    print slice
\end{verbatim}
\afterverb
%
%It is easier for the beginning programmer to look at this code and know which 
%parts are reserved words defined by Python and which parts are simply variable
%names chosen by the programmer.  It is pretty clear that Python has no fundamental
%understanding of pizza and slices and the fact that a pizza consists of a set
%of one or more slices.
 mais fcil para o programador iniciante olhar para este cdigo e saber quais 
partes so palavras reservadas definidas pelo Python e quais partes so, simplesmente, nomes de 
variveis escolhidos pelo programador.  bastante claro que o Python no tem nenhuma compreenso 
fundamental de pizza e slices e o fato de que uma pizza  constituda por um conjunto 
de um ou mais slices.

%But if our program is truly about reading data and looking for words in the data,
%{\tt pizza} and {\tt slice} are very un-mnemonic variable names.  Choosing them 
%as variable names distracts from the meaning of the program.
%
%After a pretty short period of time, you will know the most common reserved words
%and you will start to see the reserved words jumping out at you:
Mas se o nosso programa  verdadeiramente sobre a leitura de dados e a procura de palavras nos dados, 
{\tt pizza) e {\tt slice) so nomes de variveis no muito mnemnicos. Escolh-los 
como nomes de varivel, destorce o significado do programa.
%
Depois de um perodo muito curto de tempo, voc vai conhecer as palvras reservadas mais comuns,
ento voc vai comear a ver as palavras reservadas saltando em voc:
{\tt {\bf for} word {\bf in} words{\bf :}\\
\verb"    "{\bf print} word }

%The parts of the code that are defined by 
%Python ({\tt for}, {\tt in}, {\tt print}, and {\tt :}) are in bold
%and the programmer-chosen variables ({\tt word} and {\tt words}) are not in bold.  
%Many text editors are aware of Python
%syntax and will color reserved words differently to give you clues to keep 
%your variables and reserved words separate.
%After a while you will begin to read Python and quickly determine what
%is a variable and what is a reserved word.
As partes do cdigo que so definidos pelo Python ({\tt for}, {\tt in}, {\tt print}, and {\tt :}) esto em negrito 
e as variveis escolhidas pelo programador ({\tt word} and {\tt words}) no esto em negrito. 
Muitos editores de textos compreendem a sintaxe do Python 
e vo colorir palavras reservadas de forma diferente para dar a voc pistas e manter 
suas variveis e palavras reservadas separadas. 
Depois de um tempo voc comear a ler o Python e rapidamente determinar o que 
 uma varivel e o que  uma palavra reservada.


\section{Debugging}
\section{Debugando}
\index{debugging}

%At this point, the syntax error you are most likely to make is
%an illegal variable name, like {\tt class} and {\tt yield}, which
%are keywords, or \verb"odd~job" and \verb"US$", which contain
%illegal characters.
Neste ponto, o erro de sintaxe que voc est mais propenso a cometer  
um nome de varivel ilegal, como {\tt class) e {\tt yield), que 
so palavras reservadas ou \verb"odd~job" e \verb"US$", que contm 
caracteres no permitidos. 
 
\index{syntax error}
\index{error!syntax}

%If you put a space in a variable name, Python thinks it is two
%operands without an operator:
Se voc colocar um espao em um nome de varivel, o Python interpreta que so dois 
operandos sem um operador: 
\beforeverb
\begin{verbatim}
>>> bad name = 5
SyntaxError: invalid syntax
\end{verbatim}
\afterverb
%
%For syntax errors, the error messages don't help much.
%The most common messages are {\tt SyntaxError: invalid syntax} and
%{\tt SyntaxError: invalid token}, neither of which is very informative.
Para erros de sintaxe, as mensagens de erro no ajudam muito. 
As mensagens mais comuns so {\tt SyntaxError: invalid syntax} and
%{\tt SyntaxError: invalid token}, nenhuma das quais  muito informativa.

\index{error message}
\index{use before def}
\index{exception}
\index{runtime error}
\index{error!runtime}

%The runtime error you are most likely to make is a ``use before
%def;'' that is, trying to use a variable before you have assigned
%a value.  This can happen if you spell a variable name wrong:
O erro de execuo que voc est mais propensoa a cometer  ``use 
before def;'', isto , tentando usar uma varivel antes de atribuir 
um valor. Isso pode acontecer se voc digitar um nome de varivel errado:
\beforeverb
\begin{verbatim}
>>> principal = 327.68
>>> interest = principle * rate
NameError: name 'principle' is not defined
\end{verbatim}
\afterverb
%
%Variables names are case sensitive, so {\tt LaTeX} is not the
%same as {\tt latex}.
Nomes de variveis so sensveis a masculo e minsculo, desta forma, {\tt LaTeX}no  o 
mesmo que {\tt latex}.

\index{case-sensitivity, variable names}
\index{semantic error}
\index{error!semantic}

%At this point, the most likely cause of a semantic error is
%the order of operations.  For example, to evaluate $\frac{1}{2 \pi}$,
%you might be tempted to write
Neste ponto, a causa mais provvel de um erro de semntica  
a ordem das operaes. Por exemplo, para calcular $\frac{1}{2 \pi}$,
voc pode ser tentado a escrever

\beforeverb
\begin{verbatim}
>>> 1.0 / 2.0 * pi
\end{verbatim}
\afterverb
%
%But the division happens first, so you would get $\pi / 2$, which
%is not the same thing!  There is no way for Python
%to know what you meant to write, so in this case you don't
%get an error message; you just get the wrong answer.
Mas a diviso acontece primeiro, ento voc iria ficar com $\pi / 2$, que 
no  a mesma coisa! No h nenhuma maneira de o Python 
saber o que voc quis escrever, ento, neste caso voc no 
receberia uma mensagem de erro; voc apenas receberia uma resposta errada.

\index{order of operations}



%\section{Glossary}
\section{Glossrio}

\begin{description}

%\item[assignment:]  A statement that assigns a value to a variable.
\item[atribuio:] Uma declarao de que atribui um valor a uma varivel.
\index{assignment}

%\item[concatenate:]  To join two operands end to end.
concatenar: Para juntar dois operandos ponta-a-ponta.
\index{concatenation}

%\item[comment:]  Information in a program that is meant for other
%programmers (or anyone reading the source code) and has no effect on the
%execution of the program.
\item[Comentrio]: Informao em um programa que  destinado a outros 
programadores (ou qualquer pessoa lendo o cdigo fonte) e no tem qualquer efeito sobre 
a execuo do programa.
\index{comment}

%\item[evaluate:]  To simplify an expression by performing the operations
%in order to yield a single value.
\item[cacular:] simplificar uma expresso realizando as operaes, 
a fim de se obter um nico valor.

%\item[expression:]  A combination of variables, operators, and values that
%represents a single result value.
\item[expresso:] Uma combinao de variveis, operadores e valores que representa um
valor de resultado nico.
\index{expression}

%\item[floating point:] A type that represents numbers with fractional
%parts.
\item[Ponto Flutuante:] Um tipo que representa nmeros com partes 
fracionrias.
\index{floating-point}

%\item[floor division:] The operation that divides two numbers and chops off
%the fractional part.
\item[Floor Division:] A operao que divide dois nmeros e corta 
a parte fracionria.
\index{floor division}

%\item[integer:] A type that represents whole numbers.
\item[Integer:] Um tipo que representa nmeros inteiros.
\index{integer}

%\item[keyword:]  A reserved word that is used by the compiler to parse a
%program; you cannot use keywords like {\tt if}, {\tt  def}, and {\tt while} as
%variable names.
\item[Palavra Reservada:] Uma palavra reservada usada pelo compilador para analisar um 
programa; voce no pode usar palavras reservadas como {\tt if}, {\tt  def}, e {\tt while} como 
nomes de variveis.
\index{keyword}

%\item[mnemonic:] A memory aid. We often give variables mnemonic names
%to help us remember what is stored in the variable.
\item[Mnemnico:] Um auxiliar de memria. Ns, muitas vezes, damos nomes mnemnicos  variveis 
para nos ajudar lembrar o que est armazenado na mesma.
\index{mnemonic}

%\item[modulus operator:]  An operator, denoted with a percent sign
%({\tt \%}), that works on integers and yields the remainder when one
%number is divided by another.
\item[Operador mdulo:] Um operador, denotado pelo sinal de porcentagem 
%({\tt \%}), que funciona em inteiros e produz o restante quando um 
nmero  dividido por outro.
\index{modulus operator}
\index{operator!modulus}

%\item[operand:]  One of the values on which an operator operates.
Operando: Um dos valores sobre os quais um operador opera.
\index{operand}

%\item[operator:]  A special symbol that represents a simple computation like
%addition, multiplication, or string concatenation.
\item[Operador:] Um smbolo especial que representa uma calculo simples, como adio, multiplicao ou concatenao de strings.
\index{operator}

%\item[rules of precedence:]  The set of rules governing the order in which
%expressions involving multiple operators and operands are evaluated.
\item[Regras de precedncia:] O conjunto de regras que regem a ordem na qual as
expresses, envolvendo mltiplos operadores e operandos, so calculadas.
\index{rules of precedence}
\index{precedence}

%\item[statement:]  A section of code that represents a command or action.  So
%far, the statements we have seen are assignments and print statements.
\item[Declarao:] Uma seo de cdigo que representa um comando ou ao. At 
o momento, as declaraes que temos visto so de atribuies e impresso.
\index{statement}

%\item[string:] A type that represents sequences of characters.
item[String:] Um tipo que representa sequncias de caracteres.
\index{string}

%\item[type:] A category of values.  The types we have seen so far
%are integers (type {\tt int}), floating-point numbers (type {\tt
%float}), and strings (type {\tt str}).
item[Tipo:] Uma categoria de valores. Os tipos que vimos at o momento 
so inteiros (tipo {\tt int}), nmeros de ponto flutuante (tipo {\tt
%float}) e strings (tipo {\tt str}).
\index{type}

%\item[value:]  One of the basic units of data, like a number or string, 
%that a program manipulates.
\item[Valor:] Uma das unidades bsicas de dados, como um nmero ou string,
que um programa manipula.
\index{value}

%\item[variable:]  A name that refers to a value.
\item[Varivel:] Um nome que se refere a um valor.
\index{variable}

\end{description}

%\section{Exercises}
\section{Exercos}

\begin{ex}
%Write a program that uses  \verb"raw_input" to prompt a user for their name 
%and then welcomes them.
Escreva um programa que utiliza \verb"raw_input" para solicitar o nome de um usurio, 
em seguida, saud-lo.

\begin{verbatim}
Enter your name: Chuck
Hello Chuck
\end{verbatim}

\end{ex}

\begin{ex}
%Write a program to prompt the user for hours and rate per hour to compute
%gross pay.
Escreva um programa para solicitar ao usurio por, horas e taxa por hora, e ento,  calcular 
salrio bruto.
\begin{verbatim}
Enter Hours: 35
Enter Rate: 2.75
Pay: 96.25
\end{verbatim}
\end{ex}
%
%We won't worry about making sure our pay has exactly two digits after
%the decimal place for now.  If you want, you can play with the 
%built-in Python {\tt round} function to properly round the resulting pay
%to two decimal places.
No estamos preocupados em fazer com que o nosso pagamento tenha exatamente dois dgitos depois 
da vrgula, por enquanto. Se voc quiser, voc executar utilizando 
a funo interna {\tt round} do Python para adequadamente arredondar o pagamento resultante 
com duas casas decimais.

\begin{ex}
%Assume that we execute the following assignment statements:
Suponha que ns executamos as seguintes declaraes de atribuio:

\begin{verbatim}
width = 17
height = 12.0
\end{verbatim}

%For each of the following expressions, write the value of the
%expression and the type (of the value of the expression).
Para cada uma das seguintes expresses, escrever o valor da 
expresso e o tipo (do valor da expresso).

\begin{enumerate}

\item {\tt width/2}

\item {\tt width/2.0}

\item {\tt height/3}

\item {\tt 1 + 2 * 5}

\end{enumerate}

%Use the Python interpreter to check your answers.
Utilize o interpretador do Python para conferir suas respostas.
\end{ex}

\begin{ex}
%Write a program which prompts the user for a Celsius temperature,
%convert the temperature to Fahrenheit, and print out the converted
%temperature.
Escreva um programa que pede ao usurio por uma temperatura Celsius, 
converter a temperatura para Fahrenheit e imprimir a temperatura 
convertida.

\end{ex}


