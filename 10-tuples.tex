% LaTeX source for ``Python for Informatics: Exploring Information''
% Copyright (c)  2010-  Charles R. Severance, All Rights Reserved

%\chapter{Tuples}
\chapter{Tuplas}
\label{tuplechap}

%\section{Tuples are immutable}
\section{Tuplas são imutaveis}

\index{tuple}
\index{type!tuple}
\index{sequence}

%A tuple\footnote{Fun fact: The word ``tuple'' comes from the names
%given to sequences of numbers of varying lengths: single, 
%double, triple, quadruple, quituple, sextuple, septuple, etc.}
%is a sequence of values much like a list.  
%The values stored in a tuple can be any type, and
%they are indexed by integers.
%The important difference is that tuples are {\bf immutable}.
%Tuples are also {\bf comparable} and {\bf hashable} so we can 
%sort lists of them and use tuples as key values in Python
%dictionaries.
Uma tupla\footnote{Curiosidade: A palavra ``tupla'' vem de nomes 
dados a sequências de numeros de diferentes tamanhos: unico, dobro,
triplo, quadruplo, quíntuplo, séxtuplo, sétuplo, etc.} é uma sequência
de valores bem parecida com uma lista.
Os valores guardados em uma tupla podem ser de qualquer tipo, e
eles são indexados utilizando inteiros.
A diferença importante é que tuplas são {\bf imutaveis}.
Tuplas também são {\bf comparáveis} e {\bf nunca mudam} então nós
organizamos listas delas e usamos tuplas como valores em dicionários Python.

\index{mutability}
\index{hashable}
\index{comparable}
\index{immutability}

%Syntactically, a tuple is a comma-separated list of values:
Sintaticamente, uma tupla é um lista de valores separados por virgulas:

\beforeverb
\begin{verbatim}
>>> t = 'a', 'b', 'c', 'd', 'e'
\end{verbatim}
\afterverb
%
%Although it is not necessary, it is common to enclose tuples in
%parentheses to help us quickly identify tuples when we look at
%Python code:
Apesar disto não ser necessario, é comum fechar tuplas entre parênteses
para ajudar-nos á rapidamente identificar tuplas quando nós olhamos
para um codigo em Python:

\index{parentheses!tuples in}

\beforeverb
\begin{verbatim}
>>> t = ('a', 'b', 'c', 'd', 'e')
\end{verbatim}
\afterverb
%
%To create a tuple with a single element, you have to include the final
%comma:
Para criar uma tupla com um unico elemento, você deve incluir a virgula
final:

\index{singleton}
\index{tuple!singleton}

\beforeverb
\begin{verbatim}
>>> t1 = ('a',)
>>> type(t1)
<type 'tuple'>
\end{verbatim}
\afterverb
%
%Without the comma Python treats \verb"('a')" as an expression 
%with a string in parentheses that evaluates to a string:
Sem a virgula o Python irá tratar \verb"('a')" como uma expressão
com uma string entre os parênteses, assim alterando o valor para uma string:

\beforeverb
\begin{verbatim}
>>> t2 = ('a')
>>> type(t2)
<type 'str'>
\end{verbatim}
\afterverb
%
%Another way to construct a tuple is the built-in function {\tt tuple}.
%With no argument, it creates an empty tuple:
Uma outra forma de construir uma tupla é a função construtora {\tt tuple}.
Sem nenhum argumento, irá criar uma tupla vazia:

\index{tuple function}
\index{function!tuple}

\beforeverb
\begin{verbatim}
>>> t = tuple()
>>> print t
()
\end{verbatim}
\afterverb
%
%If the argument is a sequence (string, list, or tuple), the result
%of the call to {\tt tuple} is a tuple with the elements of the sequence:
Se o argumento for uma sequência (string, lista ou tupla), o resultado
da chamada da {\tt tuple} será uma tupla com os elementos em sequência:

\beforeverb
\begin{verbatim}
>>> t = tuple('lupins')
>>> print t
('l', 'u', 'p', 'i', 'n', 's')
\end{verbatim}
\afterverb
%
%Because {\tt tuple} is the name of a constructor, you should
%avoid using it as a variable name.
Por causa de {\tt tuple} ser o nome do construtor, você deve
evitar usar como nome de alguma variavel.

%Most list operators also work on tuples.  The bracket operator
%indexes an element:
A maioria dos operadores das listas também functionam nas tuplas.
Os conchetes indexam um elemento:

\index{bracket operator}
\index{operator!bracket}

\beforeverb
\begin{verbatim}
>>> t = ('a', 'b', 'c', 'd', 'e')
>>> print t[0]
'a'
\end{verbatim}
\afterverb
%
%And the slice operator selects a range of elements.
E o o operador de corte seleciona uma serie de elementos.

\index{slice operator}
\index{operator!slice}
\index{tuple!slice}
\index{slice!tuple}

\beforeverb
\begin{verbatim}
>>> print t[1:3]
('b', 'c')
\end{verbatim}
\afterverb
%
%But if you try to modify one of the elements of the tuple, you get
%an error:
Mas se você tentar modificar algum elemento da tupla, você receberá
um erro:

\index{exception!TypeError}
\index{TypeError}
\index{item assignment}
\index{assignment!item}

\beforeverb
\begin{verbatim}
>>> t[0] = 'A'
TypeError: object doesn't support item assignment
\end{verbatim}
\afterverb
%
%You can't modify the elements of a tuple, but you can replace
%one tuple with another:
Você não pode modificar os elementos de uma tupla, mas você pode
substituir uma tupla por outra:

\beforeverb
\begin{verbatim}
>>> t = ('A',) + t[1:]
>>> print t
('A', 'b', 'c', 'd', 'e')
\end{verbatim}
\afterverb
%

%\section{Comparing tuples}
\section{Comparando tuplas}

\index{comparison!tuple}
\index{tuple!comparison}
\index{sort method}
\index{method!sort}

%The comparison operators work with tuples and other sequences.
%Python starts by comparing the first element from each
%sequence.  If they are equal, it goes on to the next element,
%and so on, until it finds elements that differ.  Subsequent
%elements are not considered (even if they are really big).
Os operadores de comparação funcionam com tuplas e outras sequências.
O Python começa a comparar o primeiro elemento de cada sequência.
Se eles forem iguais, irá para o proximo elemento, e assim sucessivamente,
até encontrar um elemento seja diferente. Elementos subsequentes
não são considerados (mesmo que eles sejam muito grandes).

\beforeverb
\begin{verbatim}
>>> (0, 1, 2) < (0, 3, 4)
True
>>> (0, 1, 2000000) < (0, 3, 4)
True
\end{verbatim}
\afterverb
%
%The {\tt sort} function works the same way.  It sorts 
%primarily by first element, but in the case of a tie, it sorts
%by second element, and so on.  
A função {\tt sort} funciona da mesma forma. Ela ordena
primeiramente pelo primeiro elemento, mas no caso de um laço,
ela ordena pelo segundo elemento, e assim sucessivamente.

%This feature lends itself to a pattern called {\bf DSU} for 
Este recurso se presta a um padrão chamado {\bf DSU} para

\begin{description}

%\item[Decorate] a sequence by building a list of tuples
%with one or more sort keys preceding the elements from the sequence,
\item[Decorate]. Organiza uma sequência construindo uma lista de tuplas
com uma ou mais chaves ordenadas precedendo os elementos da sequência,

%\item[Sort] the list of tuples using the Python built-in {\tt sort}, and
\item[Sort]. Ordena a lista de tuplas utilizando o ferramenta embutida 
{\tt sort} do Python, e

%\item[Undecorate] by extracting the sorted elements of the sequence.
\item[Undecorate]. Desorganiza extraindo os elementos ordenados da sequência.

\end{description}

\label{DSU}
\index{DSU pattern}
\index{pattern!DSU}
\index{decorate-sort-undecorate pattern}
\index{pattern!decorate-sort-undecorate}
\index{Romeo and Juliet}

%For example, suppose you have a list of words and you want to
%sort them from longest to shortest:
Por exemplo, suponha que você tenha uma lista de palavras e você
quer organiza-la da palavra mais longa para a mais curta:

\beforeverb
\begin{verbatim}
txt = 'but soft what light in yonder window breaks'
words = txt.split()
t = list()
for word in words:
   t.append((len(word), word))

t.sort(reverse=True)

res = list()
for length, word in t:
    res.append(word)

print res
\end{verbatim}
\afterverb
%
%The first loop builds a list of tuples, where each
%tuple is a word preceded by its length.
O primeiro laço cria uma lista de tuplas, onde cada tupla é
uma palavra precedida pelo seu tamanho.

%{\tt sort} compares the first element, length, first, and
%only considers the second element to break ties.  The keyword argument
%{\tt reverse=True} tells {\tt sort} to go in decreasing order.
{\tt sort} compara o primeiro elemento, tamanho, em primeiro lugar, e
somente considera o segundo elemento para quebrar o laços.

\index{keyword argument}
\index{argument!keyword}
\index{traversal}

%The second loop traverses the list of tuples and builds a list of
%words in descending order of length.  The four-character words
%are sorted in {\em reverse} alphabetical order, so ``what'' appears
%before ``soft'' in the following list.
O segundo laço atravessa a lista de tuplas e constroi uma lista de 
palavras ordenadas por seu tamanho. A palavras de quatro caracteres
são organizadas no {\em inverso} da ordem alfabética, então ``what''
aparece antes de ``soft'' na lista a seguir.

%The output of the program is as follows:
A saida do programa sera a seguinte:
%
\beforeverb
\begin{verbatim}
['yonder', 'window', 'breaks', 'light', 'what', 
'soft', 'but', 'in']
\end{verbatim}
\afterverb
%
%Of course the line loses much of its poetic impact 
%when turned into a Python list and sorted in 
%descending word length order.
Claramente a linha perde muito do seu poder poético
quanto se torna uma lista do Python e é ordenada
pelo tamanho das palavras.

%\section{Tuple assignment}
%\label{tuple assignment}
\section{Declaração de tuplas}
\label{tuple assignment}

\index{tuple!assignment}
\index{assignment!tuple}
\index{swap pattern}
\index{pattern!swap}

%One of the unique syntactic features of the Python language
%is the ability to have a tuple on the left
%side of an assignment statement.  This allows you to assign
%more than one variable at a time when the left side is a 
%sequence.
Uma das características sintaticas unicas da linguagem Python é
a abilidade de ter tuplas a esquerda de uma declaração de variavel.
Isso te permite declarar mais que uma variavel por vez quando o
lado esquerdo for uma sequência.

%In this example we have a two-element list (which is a sequence) and
%assign the first and second elements of the sequence
%to the variables {\tt x} and {\tt y} in a single statement.
Nesse exemplo nós temos duas listas (que são uma sequência) e
designamos o primeiro e o segundo elemento da sequência para as
variaveis {\tt x} e {\tt y} em uma unica declaração.

\beforeverb
\begin{verbatim}
>>> m = [ 'have', 'fun' ]
>>> x, y = m
>>> x
'have'
>>> y
'fun'
>>> 
\end{verbatim}
\afterverb
%
%It is not magic, Python \emph{roughly} translates the 
%tuple assignment syntax
%to be the following:\footnote{Python does not translate the 
%syntax literally.  For example, if you try this with a dictionary,
%it will not work as might expect.}
Isso não é magica, o Python \emph{grosseiramente} traduz a
sintaxe de declaração da tupla
para ser a seguinte:\footnote{O Python não traduz a sintaxe
literalmente. Por exemplo, se você tentar isso com um dicionario, não
irá functionar como o experado.}

\beforeverb
\begin{verbatim}
>>> m = [ 'have', 'fun' ]
>>> x = m[0]
>>> y = m[1]
>>> x
'have'
>>> y
'fun'
>>> 
\end{verbatim}
\afterverb

%Stylistically when we use a tuple on the left side of the assignment
%statement, we omit the parentheses, but the following is an equally 
%valid syntax:
Sistematicamente quando nós usamos uma tupla no lado esquerdo da
declaração, nós omitimos os parenteses, mas a seguir temos uma sintaxe
igualmente valida:

\beforeverb
\begin{verbatim}
>>> m = [ 'have', 'fun' ]
>>> (x, y) = m
>>> x
'have'
>>> y
'fun'
>>> 
\end{verbatim}
\afterverb
%
%A particularly clever application of tuple assignment allows
%us to {\bf swap} the values of two variables in a single statement:
Uma aplicação particularmente inteligente de declaração de tuplas 
nos permite {\bf trocar} os valores de duas variaveis em uma unica declaração:

\beforeverb
\begin{verbatim}
>>> a, b = b, a
\end{verbatim}
\afterverb
%
%Both sides of this statement are tuples, but
%the left side is a tuple of variables; the right side is a tuple of
%expressions.  Each value on the right side 
%is assigned to its respective variable on the left side.  
%All the expressions on the right side are evaluated before any
%of the assignments.
Ambos os lados dessa declaração são tuplas, mas
a da esquerda é uma tupla de variaveis; a da direita é uma tupla de
expressões. Cada valor no lado esquerdo
é uma atribuição a respectiva variavel no lado esquerdo.
Todas as expressões no lado direito são avaliadas antes de qualquer
uma das declarações.

%The number of variables on the left and the number of
%values on the right must be the same:
O número de veriaveis do lado esquerdo e o numero de valores
no lado direito devem ser iguais:

\index{exception!ValueError}
\index{ValueError}

\beforeverb
\begin{verbatim}
>>> a, b = 1, 2, 3
ValueError: too many values to unpack
\end{verbatim}
\afterverb
%
%More generally, the right side can be any kind of sequence
%(string, list, or tuple).  For example, to split an email address
%into a user name and a domain, you could write:
Mas geralmente, o lado direito pode ser de qualquer tipo de sequência
(string, lista, ou tupla). Por exemplo, para dividir um email em
um nome de usuario e um dominio, você pode escrever:

\index{split method}
\index{method!split}
\index{email address}

\beforeverb
\begin{verbatim}
>>> addr = 'monty@python.org'
>>> uname, domain = addr.split('@')
\end{verbatim}
\afterverb
%
%The return value from {\tt split} is a list with two elements;
%the first element is assigned to {\tt uname}, the second to
%{\tt domain}.
O valor retornado de {\tt split} é uma lista com dois elementos;
o primeiro elemento é declarado para {\tt uname}, o segundo para
{\tt domain}.

\beforeverb
\begin{verbatim}
>>> print uname
monty
>>> print domain
python.org
\end{verbatim}
\afterverb
%

%\section{Dictionaries and tuples}
\section{Dicionarios e tuplas}

\index{dictionary}
\index{items method}
\index{method!items}
\index{key-value pair}

%Dictionaries have a method called {\tt items} that returns a list of
%tuples, where each tuple is a key-value 
%pair\footnote{This behavior is slightly different in Python 3.0.}.
Dicionarios tem um metodo chamado {\tt items} que retorna uma lista de
tuplas, onde cada tupla contem um par de chave-valor.
\footnote{Esse procedimento é um pouco diferente no Python 3.0.}.

\beforeverb
\begin{verbatim}
>>> d = {'a':10, 'b':1, 'c':22}
>>> t = d.items()
>>> print t
[('a', 10), ('c', 22), ('b', 1)]
\end{verbatim}
\afterverb
%
%As you should expect from a dictionary, the items are in no
%particular order.
Como você deve esperar de um dicionario, os itens estão sem
uma ordem em particular.

%However, since the list of tuples is a list, and tuples are comparable,
%we can now sort the list of tuples.  Converting a dictionary
%to a list of tuples is a way for us to output the contents of a 
%dictionary sorted by key:
Entretanto, uma vez que a lista de tuplas é uma lista, e tuplas são comparaveis,
nós agora podemos organizar a lista de tuplas. Convertento um dicionario
em uma lista de tuplas é uma forma de nós exibirmos os conteudos de um
dicionario organizado pelas chaves:

\beforeverb
\begin{verbatim}
>>> d = {'a':10, 'b':1, 'c':22}
>>> t = d.items()
>>> t
[('a', 10), ('c', 22), ('b', 1)]
>>> t.sort()
>>> t
[('a', 10), ('b', 1), ('c', 22)]
\end{verbatim}
\afterverb
%
%The new list is sorted in ascending alphabetical order by the key value.
A nova lista é organizada em ordem alfabetica pelo nome da chave.

%\section{Multiple assignment with dictionaries}
\section{Multipla declaração com dicionarios}

\index{traverse!dictionary}
\index{dictionary!traversal}

%Combining {\tt items}, tuple assignment, and {\tt for}, you
%can see a nice code pattern for traversing the keys and values of a dictionary
%in a single loop:
Combinando {\tt items}, declaração de tuplas, e o laço {\tt for}, você
pode ver um bom modelo de codigo para percorrer as chaves e valores de um
dicionario em um unico laço:

\beforeverb
\begin{verbatim}
for key, val in d.items():
    print val, key
\end{verbatim}
\afterverb
%
%This loop has two {\bf iteration variables} because {\tt items} returns
%a list of tuples and {\tt key, val} is a tuple assignment
%that successively iterates through each of the key-value pairs in the
%dictionary.  
Esse laço tem duas {\bf variaveis de iteração} pois {\tt items} retorna
uma lista de tuplas e {\tt key, val} é uma tupla de declaração
que sucessivamente itera através de cada um dos pares de chave-valor no
dicionario.

%For each iteration
%through the loop, both {\tt key} and {\tt value} are advanced to the
%next key-value pair in the dictionary (still in hash order).
Para cada iteração
através do laço, ambos {\tt key} e {\tt value} são avançados para o
proximo par de chave-valor no dicionario (continua em uma ordem hash).

%The output of this loop is:
A saida desse laço será:

\beforeverb
\begin{verbatim}
10 a
22 c
1 b
\end{verbatim}
\afterverb
%
%Again, it is in hash key order (i.e., no particular order).
Novamente, está em uma hash ordenada pela chave (i.e., nenhuma ordem em particular).

%If we combine these two techniques, we can print out the contents
%of a dictionary sorted by the \emph{value} stored in each key-value
%pair.
Se nós combinarmos essas duas tecnicas, nós podemos imprimir o conteudo
de um dicionario ordenado pelo \emph{valor} armazenado em cada par de 
chave-valor.

%To do this, we first make a list of tuples where each tuple is 
%{\tt (value, key)}.  The {\tt items} method would give us a list of 
%{\tt (key, value)} tuples---but this time we want to sort by value, not key.
%Once we have constructed the list with the value-key tuples, it is a simple
%matter to sort the list in reverse order and print out the new, sorted list.
Para fazer isso, nós primeiramente criamos uma lista de tuplas onde cada tupla é
{\tt (valor, chave)}. O metodo {\tt items} nós dará uma lista de tuplas 
{\tt (chave, valor)} ---mas agora nós queremos organizar pelos valores, não pelas chaves.
Uma vez que tenha sido construida a lista com as tuplas de chave-valor, será
simplesmente questão de organizar a lista em ordem reversa e exibir a nova 
lista organizada.

\beforeverb
\begin{verbatim}
>>> d = {'a':10, 'b':1, 'c':22}
>>> l = list()
>>> for key, val in d.items() :
...     l.append( (val, key) )
... 
>>> l
[(10, 'a'), (22, 'c'), (1, 'b')]
>>> l.sort(reverse=True)
>>> l
[(22, 'c'), (10, 'a'), (1, 'b')]
>>> 
\end{verbatim}
\afterverb
%
%By carefully constructing the list of tuples to have the value as the first
%element of each tuple, we can sort the list of tuples and get our dictionary
%contents sorted by value.
Esteja atento quando for construir a lista de tuplas para ter os valores como
primeiro elemento de cada tupla, assim nós podemos organizar as tuplas e pegar os
conteudos do dicionario organizado por valor.

%\section{The most common words}
\section{As palavras mais comuns}

\index{Romeo and Juliet}
%Coming back to our running example of the text from \emph{Romeo and Juliet} 
%Act 2, Scene 2, we can augment our program to use this technique to 
%print the ten most common words in the text as follows:
Voltando ao nosso exemplo de texto do \emph{Romeo and Juliet}
Ato 2, cena 2, nós podemos aumentar nosso programa para usar essa tecnica
para exibir as dez palavras mais comuns no texto como vocÊ pode ver a seguir:

\beforeverb
\begin{verbatim}
import string
fhand = open('romeo-full.txt')
counts = dict()
for line in fhand:
    line = line.translate(None, string.punctuation)
    line = line.lower()
    words = line.split()
    for word in words:
        if word not in counts:
            counts[word] = 1
        else:
            counts[word] += 1

# Sort the dictionary by value
lst = list()
for key, val in counts.items():
    lst.append( (val, key) )

lst.sort(reverse=True)

for key, val in lst[:10] :
    print key, val
\end{verbatim}
\afterverb
%
%The first part of the program which reads the file and computes 
%the dictionary that maps each word to the count of words in the 
%document is unchanged.  But instead of simply printing out 
%{\tt counts} and ending the program, we construct a list 
%of {\tt (val, key)} tuples and then sort the list in reverse order.
A primeira parte do programa que lê o arquivo e computa o dicionario
que mapeia cada palavra para contar as palavras no documento está
inalterado. Mas ao inves de simplesmente exibir {\tt counts} e 
finalizar o programa, nós construimos uma lista de tuplas 
{\tt (valor, chave)} e então ordenamos a lista em ordem reversa.

%Since the value is first, it will be used for the comparisons. 
%If there is more than one tuple with the same value, it will look
%at the second element (the key), so tuples where the value is the
%same will be further sorted by the alphabetical order of the key.
Uma vez que o valor seja o primeiro, ele será usado nas comparações.
Se tiver mais que uma tupla com o mesmo valor, ele ira comparar
com o segundo elemento (a chave), então em tuplas onde os valores são
os mesmos ainda serão classificadas pela ordem alfabetica das chaves.

%At the end we write a nice {\tt for} loop which does a multiple
%assignment iteration and prints out the ten most common words
%by iterating through a slice of the list ({\tt lst[:10]}).
No final nós escrevemos um laço {\tt for} que faz multiplas
iterações em declarações e exibe as dez palavras mais comuns iterando
através de uma divisão da lista ({\tt lst[:10]}).

%So now the output finally looks like what we want for our word 
%frequency analysis.
Então a saida irá parece como o que nós queremos para o nosso
analista de palavras repetidas.

\beforeverb
\begin{verbatim}
61 i
42 and
40 romeo
34 to
34 the
32 thou
32 juliet
30 that
29 my
24 thee
\end{verbatim}
\afterverb
%
%The fact that this complex data parsing and analysis 
%can be done with an easy-to-understand 19-line Python
%program is one reason why Python is a good choice as a language 
%for exploring information.
O fato que esse complexo sistema de decomposição e analise de dados
pode ser feito utilizando 19 linhas de facil compreenção no Python
é uma das razões de o Python ser uma boa escolha para
exmplorar informações.

%\section{Using tuples as keys in dictionaries}
\section{Usando tuplas como chaves em dicionarios}

\index{tuple!as key in dictionary}
\index{hashable}

%Because tuples are {\bf hashable} and lists are not, if we want to 
%create a {\bf composite} key to use in a dictionary we must use a tuple as
%the key.
Por conta de tuplas serem {\bf imutaveis} e listas não, se nós quisermos
criar uma chave {\bf composta} para usar em um dicionario nós usamos a 
tupla como chave.

%We would encounter a composite key if we wanted to create a 
%telephone directory that maps
%from last-name, first-name pairs to telephone numbers.  Assuming
%that we have defined the variables 
%{\tt last}, {\tt first}, and {\tt number}, we could write
%a dictionary assignment statement as follows:
Nós devemos encontrar uma chave composta se nós quisermos criar
uma lista telefonica que mapeia do ultimo nome, e os primeiros nomes
para os numeros de telefones. Assumindo que nós definimos as
variaveis {\tt last}, {\tt first}, e {\tt number}, nós podemos
escrever uma declaração de um dicionario assim:

\beforeverb
\begin{verbatim}
directory[last,first] = number
\end{verbatim}
\afterverb
%
%The expression in brackets is a tuple.  We could use tuple
%assignment in a {\tt for} loop to traverse this dictionary.
A expressão entres os colchetes é uma tupla. Nós podemos usar
tuplas em declarações em um laço {\tt for} para percorrer esse
dicionario.

\index{tuple!in brackets}

\beforeverb
\begin{verbatim}
for last, first in directory:
    print first, last, directory[last,first]
\end{verbatim}
\afterverb
%
%This loop traverses the keys in {\tt directory}, which are tuples.  It
%assigns the elements of each tuple to {\tt last} and {\tt first}, then
%prints the name and corresponding telephone number.
Esse laço percorre as chaves no {\tt directory}, que são tuplas. E
atribui os elementos de cada tupla para o {\tt last} e {\tt first},
então exibe o nome do número de telefone correspondente.

%\section{Sequences: strings, lists, and tuples---Oh My!}
\section{Sequências: strings, listas, e tuplas---Oh!}
\index{sequence}

%I have focused on lists of tuples, but almost all of the examples in
%this chapter also work with lists of lists, tuples of tuples, and
%tuples of lists.  To avoid enumerating the possible combinations, it
%is sometimes easier to talk about sequences of sequences.
Nós estavamos focados em listas de tuplas, mas quase todos os exemplos
deste capitulo também funcionam com listas de listas, tuplas de tuplas,
e tuplas de listas. Para evitar de numerar possiveis combinações, é
mais facil falar sobre sequências de sequências.

%In many contexts, the different kinds of sequences (strings, lists, and
%tuples) can be used interchangeably.  So how and why do you choose one
%over the others?
Em varios contextos, os diferentes tipos de sequências (strings, listas e
tuplas) podem ser usadas de forma intercambiável. Então por que você
escolheria um ao inves de outro?

\index{string}
\index{list}
\index{tuple}
\index{mutability}
\index{immutability}

%To start with the obvious, strings are more limited than other
%sequences because the elements have to be characters.  They are
%also immutable.  If you need the ability to change the characters
%in a string (as opposed to creating a new string), you might
%want to use a list of characters instead.
Para começar com o óbvio, strings são mais limitadas que outras
sequências por conta dos elementos terem que ser caracteres. Elas
também são imutaveis. Se você necessita da abilidade de mudar os
caracteres em uma string (em vez de criar uma nova string), você 
deveria usar uma lista de caracteres como alternativa.

%Lists are more common than tuples, mostly because they are mutable.
%But there are a few cases where you might prefer tuples:
Listas são mais comuns que tuplas, principalmente por serem mutaveis.
Mas tem alguns casos onde você ira preferir usar tuplas:

\begin{enumerate}

%\item In some contexts, like a {\tt return} statement, it is
%syntactically simpler to create a tuple than a list.  In other
%contexts, you might prefer a list.
\item Em alguns contextos, como uma declaração {\tt return}, será
sintaticamente mais simples criar uma tupla do que uma lista. Em 
outros contextos, você pode preferir usar uma lista.

%\item If you want to use a sequence as a dictionary key, you
%have to use an immutable type like a tuple or string.
\item Se você quiser usar uma sequência como uma chave de dicionario,
você deve usar uma do tipo imutavel, como uma tupla ou string.

%\item If you are passing a sequence as an argument to a function,
%using tuples reduces the potential for unexpected behavior
%due to aliasing.
\item Se você estiver passando uma sequência como um argumento para
uma função, utilizar tuplas reduz o potencial de ter um comportamento
inexperado devido ao aliasing.

\end{enumerate}

%Because tuples are immutable, they don't provide methods
%like {\tt sort} and {\tt reverse}, which modify existing lists.
%However Python provides the built-in functions {\tt sorted}
%and {\tt reversed}, which take any sequence as a parameter
%and return a new sequence with the same elements in a different
%order.
Por conta das tuplas serem imutaveis, elas não tem metodos como o
{\tt sort} e {\tt reverse}, os quais modificam a lista existente.
No entando o Python fornece as funções embutidas {\tt sorted}
e {\tt reversed}, as quais pegam qualquer sequência como um parametro
e retornam uma nova sequência com os mesmos elementos em uma ordem
diferente.

\index{sorted function}
\index{function!sorted}
\index{reversed function}
\index{function!reversed}


%\section{Debugging}
\section{Debugando}

\index{debugging}
\index{data structure}
\index{shape error}
\index{error!shape}

%Lists, dictionaries and tuples are known generically as {\bf data
%  structures}; in this chapter we are starting to see compound data
%structures, like lists of tuples, and dictionaries that contain tuples
%as keys and lists as values.  Compound data structures are useful, but
%they are prone to what I call {\bf shape errors}; that is, errors
%caused when a data structure has the wrong type, size, or composition,
%or perhaps you write some code and forget the shape of your data
%and introduce an error.
Listas, dicionarios e tuplas são geralmente conhecidos como {\bf estruturas
de dados}; neste capitulo nós estamos começando a ver estruturas de dados
compostas, como listas de tuplas, e dicionarios que contem tuplas como
chaves e listas como valores. Estruturas de dados compostos são uteis, mas
elas são inclinadas a erros, os quais eu chamo de {\bf erros de forma};
que são erros causados quando uma estrutura de dados contem o tipo errado,
tamanho ou composição, ou talvez você tenha escrito algum codigo e esqueceu de 
modelar seus dados e introduzir um erro.

%For example, if you are expecting a list with one integer and I
%give you a plain old integer (not in a list), it won't work.
Por exemplo, se você estiver esperando uma lista com um inteiro e eu
lhe der um unico inteiro (que não está em uma lista), não irá functionar.

%When you are debugging a program, and especially if you are
%working on a hard bug, there are four things to try:
Quando você estiver debugando um programa, e especialmente se você
estiver trabalhando em um bug dificil, tem algunas coisas que você 
deve tentar:

\begin{description}

%\item[reading:] Examine your code, read it back to yourself, and
%check that it says what you meant to say.
\item[lendo:] Examine o seu codigo, leia-o novamente para si mesmo, e 
verifique se isso fala oque você quer que queria que dissesse.

%\item[running:] Experiment by making changes and running different
%versions.  Often if you display the right thing at the right place
%in the program, the problem becomes obvious, but sometimes you have to
%spend some time to build scaffolding.
\item[rodando:] Experimente fazer alterações e rodar diferentes 
versões. Frequentemente se você exibir a coisa certa no lugar certo
no programa, o problema se tornará obvio, mas algumas vezes você deve 
gastar algum tempo construindo um Scaffold. 

%\item[ruminating:] Take some time to think!  What kind of error
%is it: syntax, runtime, semantic?  What information can you get from
%the error messages, or from the output of the program?  What kind of
%error could cause the problem you're seeing?  What did you change
%last, before the problem appeared?
\item[refletir:] Gaste um tempo pensando! Que tipo de erro é esse:
sintaxe, tempo de execução, semântico? Que informação você recebeu das
mensagens de erro, ou da saida do programa? Que tipo de erro pode causar
o problema que você está vendo? Oque você mudou por ultimo, antes
do problema aparecer?

%\item[retreating:] At some point, the best thing to do is back
%off, undoing recent changes, until you get back to a program that
%works and that you understand.  Then you can start rebuilding.
\item[retrocedendo:] Em alguma hora, a melhor coisa a se fazer é
voltar atrás, desfazer as mudanças recentes, até que você volte ao 
programa que funciona e que você conhece. Então você pode começar a
reconstruir.

\end{description}

%Beginning programmers sometimes get stuck on one of these activities
%and forget the others.  Each activity comes with its own failure
%mode.
Programadores iniciantes as vezes ficam presos em uma dessas ações
e esquecem as outras. Cada ação vem com o seu próprio fracasso. 

\index{typographical error}

%For example, reading your code might help if the problem is a
%typographical error, but not if the problem is a conceptual
%misunderstanding.  If you don't understand what your program does, you
%can read it 100 times and never see the error, because the error is in
%your head.
Por exemplo, ler o seu codigo pode ajudar a descobrir se o problema é
um erro tipográfico, mas não irá ajudar se o programa é um conceito
mal entendido.

\index{experimental debugging}

%Running experiments can help, especially if you run small, simple
%tests.  But if you run experiments without thinking or reading your
%code, you might fall into a pattern I call ``random walk programming'',
%which is the process of making random changes until the program
%does the right thing.  Needless to say, random walk programming
%can take a long time.
Rodar experimentos pode ajudar, especialmente se você rodar pequenos e
simples testes. Mas se você rodar experimentos sem pensar ou ler o seu
codigo, você pode cair em um padrão chamado ``programação aleatória'',
o qual é o processo de fazer mudanças aleatórias até o programa
fazer a coisa certa. Desnecessário dizer, programação aleatória
pode levar bastante tempo.

\index{random walk programming}
\index{development plan!random walk programming}

%You have to take time to think.  Debugging is like an
%experimental science.  You should have at least one hypothesis about
%what the problem is.  If there are two or more possibilities, try to
%think of a test that would eliminate one of them.
Você deve reservar um tempo para pensar. Debugar é como uma ciência
expiremental. Você deve ao menos uma hipótese sobre qual é 
o problema. Se você tiver duas ou mais possibilidades, tente pensar
em um teste que possa elimitar todas elas.

%Taking a break helps with the thinking.  So does talking.
%If you explain the problem to someone else (or even to yourself), you
%will sometimes find the answer before you finish asking the question.
Fazer um intervalo ajuda a penser. Assim como falar. Se você
explicar o problema para outra pessoa (ou até para si mesmo), você
irá algumas vezes encontrar a resposta antes de terminar de fazer a pergunta.

%But even the best debugging techniques will fail if there are too many
%errors, or if the code you are trying to fix is too big and
%complicated.  Sometimes the best option is to retreat, simplifying the
%program until you get to something that works and that you
%understand.
Mas as vezes a melhor tecnica para debugar irá falhar se tiver muitos
erros, ou se o codifo que você estiver tentando arrumar for muito grande
e complicado. As vezes a melhor opção é recriar, simplificar o programa
até você ter algo que funciona e que você entende.

%Beginning programmers are often reluctant to retreat because
%they can't stand to delete a line of code (even if it's wrong).
%If it makes you feel better, copy your program into another file
%before you start stripping it down.  Then you can paste the pieces
%back in a little bit at a time.
Programadores iniciantes são muitas vezes relutantes em recuar, porque
eles não conseguem lidar com deletar uma linha de codigo (mesmo que
esteja errada). Se isso the faz se sentir melhor, copie o seu programa
em outro arquivo antes de você começar a dissecar ele. Então você pode
colar as partes de volta pouco a pouco.

%Finding a hard bug requires reading, running, ruminating, and
%sometimes retreating.  If you get stuck on one of these activities,
%try the others.
Encontrar um bug dificil requer ler, rodar, refletir, e algumas vezes
recuar. Se você ficar preso em uma dessas ações, tente outras.


%\section{Glossary}
\section{Glossário}

\begin{description}

%\item[comparable:] A type where one value can be checked to see if it is
%greater than, less than, or equal to another value of the same type.
%Types which are comparable can be put in a list and sorted.
%\index{comparable}
\item[comparavel:] Um tipo onde um valor pode ser checado para ver se é
maior que, menor que, ou igual a outro valor do mesmo tipo.
Tipos que são comparaveis podem ser colocados em uma lista e ordenados.
\index{comparaveis}

%\item[data structure:] A collection of related values, often
%organized in lists, dictionaries, tuples, etc.
%\index{data structure}
\item[estrutura de dados:] Uma coleção de valores relacionados, 
frequentemente organizados em listas, dicionarios, tuplas, etc.
\index{estrutura de dados}

%\item[DSU:] Abbreviation of ``decorate-sort-undecorate'', a
%pattern that involves building a list of tuples, sorting, and
%extracting part of the result.
%\index{DSU pattern}
\item[DSU:] Abreviação de ``decorate-sort-undecorate'', um
padrão que envolve construir listas de tuplas, ordenar, e 
extrair parte do conteudo.
\index{padrão DSU}

%\item[gather:] The operation of assembling a variable-length
%argument tuple.
%\index{gather}
\item[gather:] Uma operação de definir uma tupla como argumento 
de tamanho varivael.

\item[hashable:] A type that has a hash function.  Immutable
types like integers,
floats, and strings are hashable; mutable types like lists and
dictionaries are not.
\index{hashable}

\item[scatter:] The operation of treating a sequence as a list of
arguments.
\index{scatter}

\item[shape (of a data structure):] A summary of the type,
size, and composition of a data structure.
\index{shape}

\item[singleton:] A list (or other sequence) with a single element.
\index{singleton}

\item[tuple:] An immutable sequence of elements.
\index{tuple}

\item[tuple assignment:] An assignment with a sequence on the
right side and a tuple of variables on the left.  The right
side is evaluated and then its elements are assigned to the
variables on the left.
\index{tuple assignment}
\index{assignment!tuple}

\end{description}


\section{Exercises}

\begin{ex}
Revise a previous program as follows:  Read and 
parse the ``From'' lines and pull out the 
addresses from the line.   Count the number of
messages from each person using a dictionary.

After all the data has been read, print 
the person with the most commits by creating
a list of (count, email) tuples from the 
dictionary.   Then sort the list in reverse
order and print out the person who has the most
commits.

\beforeverb
\begin{verbatim}
Sample Line:
From stephen.marquard@uct.ac.za Sat Jan  5 09:14:16 2008

Enter a file name: mbox-short.txt
cwen@iupui.edu 5

Enter a file name: mbox.txt
zqian@umich.edu 195
\end{verbatim}
\afterverb
\end{ex}
\begin{ex}
This program counts the distribution of the hour of the day for 
each of the messages. You can pull the hour from the ``From'' 
line by finding the time string and then splitting that string 
into parts using the colon character. Once you have accumulated 
the counts for each hour, print out the counts, one per line, 
sorted by hour as shown below. 
\beforeverb
\begin{verbatim}
Sample Execution:
python timeofday.py
Enter a file name: mbox-short.txt
04 3
06 1
07 1
09 2
10 3
11 6
14 1
15 2
16 4
17 2
18 1
19 1
\end{verbatim}
\afterverb
\end{ex}


\begin{ex}
Write a program that reads a file and 
prints the {\em letters} in decreasing order of frequency.  Your program
should convert all the input to lower case and only count the letters a-z.
Your program should not count spaces, digits, punctuation, or anything 
other than the letters a-z.
Find text samples from several different languages and see how letter frequency
varies between languages.  Compare your results with the tables at
\url{wikipedia.org/wiki/Letter_frequencies}.

\index{letter frequency}
\index{frequency!letter}

\end{ex}

