% LaTeX source for ``Python for Informatics: Exploring Information''
% Copyright (c)  2010-  Charles R. Severance, All Rights Reserved

%\chapter{Iteration}
%\index{iteration}
\chapter{Interação}
\index{interação}

%\section{Updating variables}
%\label{update}
\section{Atualizando variáveis}
\label{atualizar}

%\index{update}
%\index{variable!updating}
\index{atualizar}
\index{varivável!atualizando}

% common pattern in assignment statements is an assignment statement
%that updates a variable -- 
%where the new value of the variable depends on the old.

Um padrão comum nas instruções de atribuição é uma instrução de atribuição
que atualiza uma variável -- onde o novo valor da variável depende da antiga.

\beforeverb
\begin{verbatim}
x = x+1
\end{verbatim}
\afterverb
%
%This means ``get the current value of {\tt x}, add 1, and then
%update {\tt x} with the new value.''

%
Isto significa ``pega o valor atual de {\tt x}, adicione 1, e depois atualize
{\tt x} com o novo valor.''

%If you try to update a variable that doesn't exist, you get an
%error, because Python evaluates the right side before it assigns
%a value to {\tt x}:

Se você tentar atualizar uma variável que não existe, você receberá um erro,
porque Python avalia o lado direito antes de atribuir um valor a {\tt x}:

\beforeverb
\begin{verbatim}
>>> x = x+1
NameError: name 'x' is not defined
\end{verbatim}
\afterverb
%
%Before you can update a variable, you have to {\bf initialize}
%it, usually with a simple assignment:

Antes de você atualizar uma variável, é necessário {\bf inicializá-la},
usualmente com uma simples atribuição:

%\index{initialization (before update)}
\index{inicialização (antes de atualizar)}

\beforeverb
\begin{verbatim}
>>> x = 0
>>> x = x+1
\end{verbatim}
\afterverb
%
%Updating a variable by adding 1 is called an {\bf increment};
%subtracting 1 is called a {\bf decrement}.

%
Atualizando uma variável, adicionando 1, é o que chamamos {\bf incremento};
subtraindo 1 é o que chamamos de {\bf decremento}.

%\index{increment}
%\index{decrement}
\index{incremento}
\index{drecremento}

%\section{The {\tt while} statement}
\section{A condição {\tt while}}

%\index{statement!while}
%\index{while loop}
%\index{loop!while}
%\index{iteration}

\index{condição!while}
\index{laço while}
\index{laço!while}
\index{interação}

%Computers are often used to automate repetitive tasks.  Repeating
%identical or similar tasks without making errors is something that
%computers do well and people do poorly.
%Because iteration is so common, Python provides several
%language features to make it easier.  

Computadores são normalmente utilizado para automatizar tarefas repetitivas.
A repetição de tarefas identicas ou similares sem produzir erros é algo
computadores fazem bem e pessoal não muito. Pelo fato de iterações serem tão
comuns, Python disponibiliza muitas funcionalidades para tornar isto fácil.

%One form of iteration in Python is the {\tt while} statement.  Here is 
%a simple program that counts down from five and then says ``Blastoff!''.

Uma das formas de iterações em Python é a declaração {\tt while}. Aqui está
um programa simples que realiza uma contagem regressiva a partir de cinco e
depois diz ``Blastoff!''.

\beforeverb
\begin{verbatim}
n = 5
while n > 0:
    print n
    n = n-1
print 'Blastoff!'
\end{verbatim}
\afterverb
%
%You can almost read the {\tt while} statement as if it were English.
%It means, ``While {\tt n} is greater than 0,
%display the value of {\tt n} and then reduce the value of
%{\tt n} by 1.  When you get to 0, exit the {\tt while} statement and
%display the word {\tt Blastoff!}''

%
Você quase pode ler a declaração {\tt while} como se ela fosse escrita em
Inglês. Ou seja, ``Enquanto {\tt n} for maior que 0, mostre o valor de {\tt n}
e então subtraia o valor de {\tt n} em 1. Quando chegar ao 0, saia da
declaração do {\tt while} e mostra a palavra {\tt Blastoff!}''.

%\index{flow of execution}
\index{fluxo de execução}

%More formally, here is the flow of execution for a {\tt while} statement:
%Formalmente, este é o fluxo de execução de uma declaração {\tt while}:

\begin{enumerate}

%\item Evaluate the condition, yielding {\tt True} or {\tt False}.
\item Avalia a condição, produzindo {\tt True} ou {\tt False}.

%\item If the condition is false, exit the {\tt while} statement
%and continue execution at the next statement.
\item Se a condição for falsa, sai da declaração do {\tt while} e continuar
	a execução para a próxima declaração.

%\item If the condition is true, execute the
%body and then go back to step 1.
\item Se a condição for verdadeira, executa o corpo do {\tt while} e depois
	volta para o passo 1.

\end{enumerate}

%This type of flow is called a {\bf loop} because the third step
%loops back around to the top.  We call each time we execute the body of 
%the loop an {\bf iteration}.  For the above loop, we 
%would say, ``It had five iterations'', which means that the body of
%the loop was executed five times.

Este tipo de fluxo é chamado de {\bf laço} ({\it loop}) devido ao terceiro
passo que retorna para o início da declaraçã. Chamamos cada vez que executamos
o corpo do laço de {\bf iteração}. Para o laço anterior, podemos dizer que,
``tem cinco iterações'', que significa que o corpo do laço será executado
cinco vezes.

%\index{condition}
%\index{loop}
%\index{body}
\index{condição}
\index{laço}
\index{corpo}

%The body of the loop should change the value of one or more variables
%so that eventually the condition becomes false and the loop
%terminates.  
%We call the variable that changes each time the loop
%executes and controls when the loop finishes the 
%{\bf iteration variable}.
%If there is no iteration variable, the loop will repeat forever, 
%resulting in an {\bf infinite loop}.  

O corpo do laço deve mudar o valor de uma ou mais variáveis para que a
condição eventualmente se torne fals e o laço termine. Podemos chamar a
variável que muda a cada vez que o laço executa e controla quando ele irá
terminar de {\bf variável de iteração}. Se não houver variável de iteração,
o laço irá se repetir para sempre, resultando em um {\bf laço infinito}.

%\section{Infinite loops}
\section{Laços infinitos}

%An endless source of amusement for 
%programmers is the observation that the directions on shampoo,
%``Lather, rinse, repeat,'' are an infinite loop because 
%there is no {\bf iteration variable} telling you how many times
%to execute the loop.

Um recurso interminável de diversão para programadores é a observação do ato
de se ensaboar, ``ensaboe, enxague e repita'', é um laço infinito porque não
há variável de iteração dizendo quantas vezes o laço deve ser executado.

%\index{infinite loop}
%\index{loop!infinite}
\index{laço infinito}
\index{laço!infinito}

%In the case of {\tt countdown}, we can prove that the loop
%terminates because we know that the value of {\tt n} is finite, and we
%can see that the value of {\tt n} gets smaller each time through the
%loop, so eventually we have to get to 0.  Other times a loop is obviously
%infinite because it has no iteration variable at all.

No caso de {\tt contagem regressiva}, nós provamos que o laço terminou porque
sabemos que o valor de {\tt n} é finito, e podemos ver que o valor de {\tt n}
diminui cada vez que passa pelo laço, então eventualmente nós teremos 0. Em
outros casos o laço é obviamente infinito porque não tem variável de iteração.

%\section{``Infinite loops'' and {\tt break}}
%\index{break statement}
%\index{statement!break}

\section{``Laços infinitos'' e {\tt break}}
\index{declaração break}
\index{declaração!break}

%Sometimes you don't know it's time to end a loop until you get half
%way through the body.  In that case you can write an infinite loop on purpose
%and then use the {\tt break} statement to jump out of the loop.

Algumas vezes você não sabe se é hora de acabar o laço até que você percorra
metade do corpo. Neste caso você pode escrever um laço infinito de propósito
e então usar a declaração {\tt break} para sair do laço.

%This loop is obviously an {\bf infinite loop} because the logical 
%expression on the
%{\tt while} statement is simply the logical constant {\tt True}:

Este laço é obviamente um {\bf laço infinito} porque a expressão lógica do
{\tt while} é a constante lógica {\tt True}

\beforeverb
\begin{verbatim}
n = 10
while True:
    print n, 
    n = n - 1
print 'Done!'
\end{verbatim}
\afterverb
%
%If you make the mistake and run this code, you will learn quickly how
%to stop a runaway Python process on your system or find where the power-off
%button is on your computer.  
%This program will 
%run forever or until your battery runs out 
%because the logical expression at the top of the loop 
%is always true by virtue of the fact that the expression is the 
%constant value {\tt True}.

%
Se você cometer o erro e executar este código, você aprenderá rapidamente como
parar um processo Python no seu sistema ou onde está o botão de desligar do
seu computador. Este programa executará eternamente ou até que sua bateria
acabe por que a expressão lógica no início do laço será sempre verdadeiro
em virtude do fato que a expressão é o valor constante {\tt True}.

%While this is a dysfunctional infinite loop, we can still use this pattern
%to build useful loops as long as we carefully add code to the 
%body of the loop to explicitly exit the loop using {\tt break} 
%when we have reached 
%the exit condition.

Enquanto este laço é um laço infinito disfuncional, nós continuamos utilizando
este padão para construir laços úteis desde que adicionemos código de forma
cuidadosa no corpo do laço para explicitamente sair do laço utilizando 
{\tt break} quando alcançarmos a condição de saída.

%For example, suppose you want to take input from the user until they
%type {\tt done}.  You could write:

Por exemplo, suponha que você queira obter a entrar do usuário, até que ele
digite {\tt done}. Podemos escrver:

\beforeverb
\begin{verbatim}
while True:
    line = raw_input('> ')
    if line == 'done':
        break
    print line
print 'Done!'
\end{verbatim}
\afterverb
%
%The loop condition is {\tt True}, which is always true, so the
%loop runs repeatedly until it hits the break statement.

%
A condição do laço é {\tt True}, ou seja, é sempre verdade, então o laço
executará de forma repetida até que chegue a declaração do {\tt break}.

%Each time through, it prompts the user with an angle bracket.
%If the user types {\tt done}, the {\tt break} statement exits
%the loop.  Otherwise the program echoes whatever the user types
%and goes back to the top of the loop.  Here's a sample run:

Em cada vezes, pergunta-se ao usuário com um sinal de menor. Se o usuário
digitar {\tt done}, a declaração {\tt break} sai do laço. Caso contrário, o
programa irá imprimir qualquer coisa que o usuáro digitar e retornar para o
início do laço. Veja um exemplo:

\beforeverb
\begin{verbatim}
> hello there
hello there
> finished
finished
> done
Done!
\end{verbatim}
\afterverb
%
%This way of writing {\tt while} loops is common because you
%can check the condition anywhere in the loop (not just at the
%top) and you can express the stop condition affirmatively
%(``stop when this happens'') rather than negatively (``keep going
%until that happens.'').

%
Esta forma de escrever um laço {\tt while} é muito comum, porque você pode
verificar a condição em qualquer lugar do laço (não somente no início) e
pode definir explicitamente a condição de parar (``pare quando isto acontecer'')
contrário de negativamente (``continue até que isto aconteça.'').

%\section{Finishing iterations with {\tt continue}}
%\index{continue statement}
%\index{statement!continue}

\section{Terminando as iterações com {\tt continue}}
\index{declaração continue}
\index{declaração!continue}

%Sometimes you are in an iteration of a loop and want to finish the
%current iteration and immediately jump to the next iteration.
%In that case you can use the {\tt continue}
%statement to skip to the next iteration without finishing the
%body of the loop for the current iteration.

Algumas vezes você está em uma iteração de um laço e quiser acabar a iteração
atual e pular para a próxima iteração. Neste caso você pode utilizar a
declaração {\tt continue} para passar para a próxima iteração sem terminar o
corpo do laço da iteração atual.

%Here is an example of a loop that copies its input until the user
%types ``done'', but treats lines that start with the hash character
%as lines not to be printed (kind of like Python comments).

Aqui temos um exemplo de um laço que copia sua entrada até que o usuário
digite ``done'', mas trata a linha que inicia com um caractere cerquilha
{\tt #} como linha para não ser impressa (como um comentário em Python).

\beforeverb
\begin{verbatim}
while True:
    line = raw_input('> ')
    if line[0] == '#' :
        continue
    if line == 'done':
        break
    print line
print 'Done!'
\end{verbatim}
\afterverb
%
%Here is a sample run of this new program with {\tt continue} added.

%
Aqui temos um exemplo deste novo programa com o uso do {\tt continue}.

\beforeverb
\begin{verbatim}
> hello there
hello there
> # don't print this
> print this!
print this!
> done
Done!
\end{verbatim}
\afterverb
%
%All the lines are printed except the one that starts with the hash
%sign because when the {\tt continue} is executed, it ends 
%the current iteration and jumps
%back to the {\tt while} statement to start the next iteration, thus 
%skipping the {\tt print} statement.

%
Todas as linhas serão impressas excepto aquela que inicia com o sinal de
cerquilha porque quando o {\tt continue} é executado, ele termina a iteração
atual e pula de volta para a declaração {\tt while} para começar uma nova
iteração, mas passando a declaração {\tt print}.

%\section{Definite loops using {\tt for} }
%\index{for statement}
%\index{statement!for}
\section{Usando {\tt for} para laços}
\index{declaração for}
\index{declaração!for}

%Sometimes we want to loop through a {\bf set} of things such 
%as a list of words, the lines in a file, or a list of numbers.
%When we have a list of things to loop through, we can
%construct a \emph{definite} loop using a {\tt for} statement.
%We call the {\tt while} statement an \emph{indefinite} loop
%because it simply loops until some condition becomes {\tt False}, 
%whereas the {\tt for} loop is looping through a known
%set of items so it runs through as many iterations as there
%are items in the set.

Algumas vezes queremos que um laço passe por um {\bf conjunto} de coisas
como uma lista de palavras, as linhas e um arquivo, ou uma lista de números.
Quando temos uma lista de coisas para percorrer, construímos um laço
\emph{limitado} utilizando a declaração {\tt for}. Nós chamamos uma declaração
{\tt while} como um laço {\tt ilimitado} por que o laço executa até que alguma
condição se torne {\tt False}, enquanto o laço {\tt for} é executado em um
conjunto de itens conhecidos, então ele executa quantas iterações forem a
quantidade de itens do conjunto.

%The syntax of a {\tt for} loop is similar to the {\tt while} loop
%in that there is a {\tt for} statement and a loop body:

A sintaxe do laço {\tt for} é similar ao do {\tt while} em que há uma
declaração {\tt for} e um corpo para o laço percorrer:

\beforeverb
\begin{verbatim}
friends = ['Joseph', 'Glenn', 'Sally']
for friend in friends:
    print 'Happy New Year:', friend
print 'Done!'
\end{verbatim}
\afterverb
%
%In Python terms, 
%the variable {\tt friends} is a list\footnote{We will 
%examine lists in more detail in a later chapter.} 
%of three strings and the {\tt for}
%loop goes through the list and executes the body once
%for each of the three strings in the list resulting in this output:

%
Em Python, a variável {\tt friends} é uma lista\footnote{Nós analizaremos as
	listas em mais detalhes e um capítulo mais adiante.} de três strings e o
laço {\tt for} passa através da lista e executa o corpo uma vez para cada uma
das três strings na lista, resultando na saída:

\beforeverb
\begin{verbatim}
Happy New Year: Joseph
Happy New Year: Glenn
Happy New Year: Sally
Done!
\end{verbatim}
\afterverb
%

%Translating this {\tt for} loop to English is not as direct as the 
%{\tt while}, but if you think of friends as a {\bf set},
%it goes like this: ``Run the statements in the body of the 
%for loop once for each friend \emph{in} the set named friends.''

Traduzindo este laço {\tt for} para o Português, não é tão direto como o laço
{\tt while}, mas se você pensar em amigos como um {\bf conjunto}, fica
parecido com isto: ``Execute a declaração no corpo do laço for uma vez para
cada amigo \emph{nos} nomes dos amigos''.

%Looking at the {\tt for} loop, {\bf for} and {\bf in} are reserved
%Python keywords, and {\tt friend} and {\tt friends} are variables.

Olhando ao laço {\tt for}, {\bf for} e {\bf in} são palavras reservadas do
Python, e {\tt friend} e {\tt friends} são variáveis.

{\tt {\bf for} friend {\bf in} friends{\bf :}\\
	\verb"   "{\bf print} 'Happy New Year', friend }

%In particular, {\tt friend} is the {\bf iteration variable} for 
%the for loop.  The variable {\tt friend} changes for each iteration of
%the loop and controls when the {\tt for} loop completes.  The 
%{\bf iteration variable} steps successively through the 
%three strings stored in the {\tt friends} variable.

Em particular, {\tt friend} é a {\bf variável de iteração} do laço {\tt for}.
A variável {\tt friend} muda para cada iteração do laço e controla quando o
laço {\tt for} completa. A {\bf variável de iteração} passa sucessivamente
através das três strings armazenadas na variável {\tt friends}.


%\section{Loop patterns}
\section{Laços padrões}

%Often we use a {\tt for} or {\tt while} loop to go through a list of items
%or the contents of a file and we are looking for something such as
%the largest or smallest value of the data we scan through.

Normalmente nós utilzamos os laços {\tt for} ou {\tt while} para percorrer
uma lista de itens ou o conteúdo de um arquivo procurando por alguma coisa
como o maior ou menor valor do dado que estamos percorrendo.

%These loops are generally constructed by:

Estes laços são normalmente construidos da seguinte forma:

\begin{itemize}

%\item Initializing one or more variables before the loop starts
\item Inicializando uma ou mais variáveis antes de iniciar o laço

%\item Performing some computation on each item in the loop body, 
%possibly changing the variables in the body of the loop
\item Realizando alguma verificação em cada item no corpo do laço,
	possivelmente mudando as variáveis no corpo do laço

%\item Looking at the resulting variables when the loop completes
\item Olhando o resultado das variáveis quando o laço finaliza
\end{itemize}

%We will use a list of numbers to demonstrate the concepts and construction
%of these loop patterns.  

Utilizamos uma lista de números para demonstrar os conceitos e os padrões
para construção de laços.

%\subsection{Counting and summing loops}
\subsection{Contando e somando laços}

%For example, to count the number of items
%in a list, we would write the following {\tt for} loop:

Por exemplo, para contar o números de ítens em uma lista, podemos escrever
o seguinte laço {\tt for}:

\beforeverb
\begin{verbatim}
count = 0
for itervar in [3, 41, 12, 9, 74, 15]:
    count = count + 1
print 'Count: ', count
\end{verbatim}
\afterverb
%
%We set the variable {\tt count} to zero before the loop starts,
%then we write a {\tt for} loop to run through the list of numbers.
%Our {\bf iteration} variable is named {\tt itervar} and while we do
%not use {\tt itervar} in the loop, it does control the loop and cause
%the loop body to be executed once for each of the values in the list.

%
Nós definimos a variável {\tt count} em zero antes do laço iniciar, então
escrevemos um laço {\tt for} para percorrer uma lista de números. Nossa
variável de iteração é chamada de {\tt itervar} e enquanto não usamos a
variável {\tt itervar} no laço, ele controla o laço que o será executado
somente uma vez para cada valor na lista.

%In the body of the loop, we add 1 to the current value of {\tt count}
%for each of the values in the list.  While the loop is executing, the 
%value of {\tt count} is the number of values we have seen ``so far''.

No corpo do laço, adicionamos 1 ao valor atual de {\tt count} para cada valor
da lista. Enquanto o laço é executado, o valor da variável {\tt count} é o
número de valores que nós vimos ``até agor''.

%Once the loop completes, the value of {\tt count} is the total number
%of items.   The total number ``falls in our lap'' at the end of the 
%loop.  We construct the loop so that we have what we want when the loop
%finishes.

Uma vez que o laço termine, o valor de {\tt count} é o total de ítens. O
total de itens ``cai no seu colo'' no final do laço. Construimos o laço para
que tenhamos o que queremos quando o laço terminar.

%Another similar loop that computes the total of a set of numbers
%is as follows:

Outro laço similar que calcula o total de um conjunto de números pode ser
visto a seguir:

\beforeverb
\begin{verbatim}
total = 0
for itervar in [3, 41, 12, 9, 74, 15]:
    total = total + itervar
print 'Total: ', total
\end{verbatim}
\afterverb
%
%In this loop we \emph{do} use the {\bf iteration variable}.
%Instead of simply adding one to the {\tt count} as in the previous loop, 
%we add the actual number (3, 41, 12, etc.) to the running 
%total during each loop iteration.
%If you think about the variable {\tt total}, it contains the 
%``running total of the values so far''.  So before the loop
%starts {\tt total} is zero because we have not yet seen any values,
%during the loop {\tt total} is the running total, and at the end of 
%the loop {\tt total} is the overall total of all the values 
%in the list.

No laço nós \emph{fazemos} uso de {\bf variável de iteração}. Ao invés de
simplesmente adicionar um a variável {\tt count} como vimos no laço anterior,
nós adicionamos o número atual (3, 41, 12, etc.) ao total atual na iteração
de cada vez que o laço é executado. Se vocẽ pensar sobre a variável
{\tt total}, ela contém o ``o total dos valore até então''. Então, antes do
laço iniciar o {\tt total} é zero porque nós não vimos nenhum valor, e durante
o laço o valor de {\tt total} é o total atual, e no final do laço, {tt total}
é a soma total de todos os valores na lista.

%As the loop executes, {\tt total} accumulates the sum of the
%elements; a variable used this way is sometimes called an
%{\bf accumulator}.
%\index{accumulator!sum}

Enquanto o laço é executado, {\tt total} acumula a soma dos elementos; uma
variável utilizada desta maneira é chamada de {\bf acumulador}.
\index{acumulador!soma}

%Neither the counting loop nor the summing loop are particularly 
%useful in practice because there are built-in functions 
%{\tt len()} and {\tt sum()} that compute the number of 
%items in a list and the total of the items in the list
%respectively.

Nem o laço contador ou o laço somador são particularmentes úteis na prática
porque Python tem funções nativas {\tt len()} e {\tt sum()} que calcula o
número e o total de itens em uma lista, respectivamente.

%\subsection{Maximum and minimum loops}
\subsection{Laços de máximos e mínimos}

%\index{loop!maximum}
%\index{loop!minimum}
%\index{None special value}
%\index{special value!None}
%\label{maximumloop}
\index{laço!máximo}
\index{laço!mínio}
\index{Nenhum valor especial}
\index{Valor especial!nenhum}
\label{maximumloop}

%To find the largest value in a list or sequence, we construct the
%following loop:

Para encontrar o maior valor em uma lista ou sequẽncia, construimos o seguinte
laço:

\beforeverb
\begin{verbatim}
largest = None
print 'Before:', largest
for itervar in [3, 41, 12, 9, 74, 15]:
    if largest is None or itervar > largest :
        largest = itervar
    print 'Loop:', itervar, largest
print 'Largest:', largest
\end{verbatim}
\afterverb
%
%When the program executes, the output is as follows:
%
Ao executar o programa, a saída é a sequinte:
\beforeverb
\begin{verbatim}
Before: None
Loop: 3 3
Loop: 41 41
Loop: 12 41
Loop: 9 41
Loop: 74 74
Loop: 15 74
Largest: 74
\end{verbatim}
\afterverb
%
%The variable {\tt largest} is best thought of as 
%the ``largest value we have seen so far''.
%Before the loop, we set {\tt largest} to the constant {\tt None}.  
%{\tt None} is a special constant value which we can 
%store in a variable to mark 
%the variable as ``empty''.  
%
A variável {\tt largest} é visto como o ``maior valor que temos''. Antes do
laço nós definimos {\tt largest} com a constante {\tt None}. {\tt None} é um
valor especial que podemos utilizar em uma variável para definir esta
variável como ``vazia''.

%Before the loop starts, the largest value we have seen so far 
%is {\tt None} since we have not yet seen any values.  While the 
%loop is executing, if {\tt largest} is {\tt None} then we take
%the first value we see as the largest so far.   You can see in 
%the first iteration when the value of {\tt itervar} is 3,
%since {\tt largest} is {\tt None}, we immediately set 
%{\tt largest} to be 3.

Antes que o laço inicia, o maior valor que temos até então é {\tt None}, uma
vez que nós ainda não temos valo nenhum. Enquanto o laço está executndo, se
{\tt largest} é {\tt None} então nós pegamos o primeiro valor que temos como
o maior. Você pode ver na primeira iteração quando o valor de {\tt itervar} é
3, uma vez que {\tt largest} é {\tt None}, nós imediatamente definimos a
variável {\tt largest} para 3.

%After the first iteration, {\tt largest} is no longer {\tt None},
%so the second part of the compound logical expression that checks
%{\tt itervar > largest} triggers only when we see a value that is
%larger than the ``largest so far''.  When we see a new ``even larger''
%value we take that new value for {\tt largest}.  You can see in the 
%program output that {\tt largest} progresses from 3 to 41 to 74.

Depois da primeira iteração, {\tt largest} não é mais {\tt None}, então a
segunda parte composta da expressão lógica que verifica o gatilho
{\tt itervar > largest} somente quando o valo é maior que o ``maior até agora''.
Quando temos um novo valor ``ainda maior'' nós pegamos este novo valor e
definimos como {\tt largest}. Você pode ver na saída do programa o
progresso do {\tt largest} de 3 para 41 para 74.

%At the end of the loop, we have scanned all of the values and
%the variable {\tt largest} now does contain the largest value
%in the list.

No final do laço, nós analizamos todos os valores e a variável {\tt largest}
agora contém o maior valor na lista.

%To compute the smallest number, the code is very similar with one
%small change:

Para calcular o menor número, o código é muito similar com pequenas
diferenças:

\beforeverb
\begin{verbatim}
smallest = None
print 'Before:', smallest
for itervar in [3, 41, 12, 9, 74, 15]:
    if smallest is None or itervar < smallest:
        smallest = itervar
    print 'Loop:', itervar, smallest
print 'Smallest:', smallest
\end{verbatim}
\afterverb
%
%Again, {\tt smallest} is the ``smallest so far'' before, during, and after the 
%loop executes.  When the loop has completed, {\tt smallest} contains the
%minimum value in the list.

Novamente, {\tt smallest} é o ``menor até agora'' antes, durante e depois do
laço ser executado. Quando o laço se completa, {\tt smallest} contém o mínimo
valor na lista.

%Again as in counting and summing, the built-in functions 
%{\tt max()} and {\tt min()} make writing these exact loops
%unnecessary.

De novo, assim como contagem e soma, as funções nativas {\tt max()} e
{\tt min()} tornam estes laços desnecessários.

%The following is a simple version of the Python built-in
%{\tt min()} function:

A seguir uma versão simples da função nativa {\tt min()} do Python:

\beforeverb
\begin{verbatim}
def min(values):
    smallest = None
    for value in values:
        if smallest is None or value < smallest:
            smallest = value
    return smallest
\end{verbatim}
\afterverb
%
%In the function version of the smallest code, we removed all of the 
%{\tt print} statements so as to be equivalent to the {\tt min} 
%function which is already built in to Python.

%
Nesta pequena versão da função, retiramos todos as declarações de {\tt print}
para que fosse equivalente a função {\tt min} que é nativa no Python.

%\section{Debugging}
\section{Depurando}

%As you start writing bigger programs, you might find yourself
%spending more time debugging.  More code means more chances to
%make an error and more places for bugs to hide.

Assim que você começar a escrever programas maiores, você se encontrará
gastando mais tempos depurando. Mais códigos significam mais chances de fazer
mais erros e mais bugs para se esconder.

%\index{debugging!by bisection}
%\index{bisection, debugging by}
\index{depurando!por biseção}
\index{biseção, depuração por}

%One way to cut your debugging time is ``debugging by bisection.''
%For example, if there are 100 lines in your program and you
%check them one at a time, it would take 100 steps.

Uma forma de diminuir o tempo de depuração é ``depuração por biseção''. Por
exemplo, se você tiver 100 linhas em seu programa e você verificá-la uma por
vez, isto levaria 100 passos.

%Instead, try to break the problem in half.  Look at the middle
%of the program, or near it, for an intermediate value you
%can check.  Add a {\tt print} statement (or something else
%that has a verifiable effect) and run the program.

Ao invés disto, tente quebrar o programa pela metade. Olhe para a metade do
programa, ou próximo dele, por um valor intermediário que você possa verificar.
Adicione a declaração de {\tt print} (ou alguma coisa que tenha um efeito
verificável) e execute o programa.

%If the mid-point check is incorrect, the problem must be in the
%first half of the program.  If it is correct, the problem is
%in the second half.

Se a verificação do ponto intermediário estiver incorret, o problema pode
estar na primeira metade do programa. Se estiver correto, o problema está
na segunda parte.

%Every time you perform a check like this, you halve the number
%of lines you have to search.  After six steps (which is much
%less than 100), you would be down to one or two lines of code,
%at least in theory.

Toda vez que você executar uma verificação como esta, você reduzirá o número
de linha que você tem que procurar. Depois de seis passos (o que é muita menos
que 100), você poderia diminuir para uma ou duas linha de código, pelo menos
em teoria.

%In practice it is not always clear what
%the ``middle of the program'' is and not always possible to
%check it.  It doesn't make sense to count lines and find the
%exact midpoint.  Instead, think about places
%in the program where there might be errors and places where it
%is easy to put a check.  Then choose a spot where you
%think the chances are about the same that the bug is before
%or after the check.

Na prática nem sempre está claro qual é a ``metade do programa'' e nem sempre
é possível verificar. Não faz sentido contar as linhas e achar exatamente o
meio do programa. Ao contrário, pense sobre lugares no programa onde podem
haver erros e lugares onde é fácil colocar uma verificação, (um {\tt print})
Então escolha um lugar onde você acha que pode ocorrer erros e faça uma
verificação antes e depois a procurar de erros.

%\section{Glossary}
\section{Glossário}

\begin{description}

%\item[accumulator:] A variable used in a loop to add up or
%accumulate a result.
%\index{accumulator}
\item[acumulador:] Uma variável utilizada em um laço para adicionar e
	acumular o resultado.
	\index{acumulador}

%\item[counter:] A variable used in a loop to count the number
%of times something happened.  We initialize a counter to 
%zero and then increment the counter each time we want to
%``count'' something.
%\index{counter}

\item[contador:] Uma variável utilizada em um laço para contar um número
	de vezes que uma coisa aconteça. Nós inicializamos o contador em zero e
	depois incrementamos o contador cada vez que quisermos ``contar'' alguma
	coisa.
	\index{contador}

%\item[decrement:] An update that decreases the value of a variable.
%\index{decrement}

\item[decremento:] Uma atualização que diminui o valor de uma variável.
	\index{decremento}

%\item[initialize:] An assignment that gives an initial value to
%a variable that will be updated.

\item[inicializador:] Uma atribuição que dá um valor inicial para a variável
	que será atualizada.

%\item[increment:] An update that increases the value of a variable
%(often by one).
%\index{increment}

\ite[incremento:] Uma atualização que aumenta o valor de uma varivel (muitas
vezes por um).
\index{incremento}

%\item[infinite loop:] A loop in which the terminating condition is
%never satisfied or for which there is no terminating condition.
%\index{infinite loop}

\item[laço infinito:] Um laço onde a condição terminal nunca é satisfeita ou
	que não hava condição terminal.
	\index{laço infinito}

%\item[iteration:] Repeated execution of a set of statements using
%either a function that calls itself or a loop.
%\index{iteration}
\item[iteração:] Execução repetida de um conjunto de declarações utilizando
	uma função ou um laço que se executa.
	\index{iteração}

\end{description}


%\section{Exercises}
\section{Exercícios}

\begin{ex}
%Write a program which repeatedly reads numbers until the user
%enters ``done''.
%Once ``done'' is entered, print out the total, count, and average
%of the numbers.  If the user enters anything other than a number, 
%detect their mistake using {\tt try} and {\tt except} and 
%print an error message and skip to the next number.

Escreva um programa que repetidamente leia um número até que o usuário digite
``done''. Uma vez que ``done'' é digitada, imprima o total, soma e a média dos
números. Se o usuário digitar qualquer coisa diferente de um número, detecte
o engano utilizando {\tt try} e {\tt except} e imprima uma mensagem de erro e
passe para o próximo número.

\begin{verbatim}
Enter a number: 4
Enter a number: 5
Enter a number: bad data
Invalid input
Enter a number: 7
Enter a number: done
16 3 5.33333333333
\end{verbatim}
\end{ex}

\begin{ex}
%Write another program that prompts for a list of numbers as above
%and at the end prints out both the maximum and minimum of the numbers instead of the average.

Escreva outro programa que solicita uma lista de números, como acima, e no
final imprima o máximo e o mínimo dos números ao invés da média.
\end{ex}


